\noindent Il routing è realizzato dal framework Express e le richieste HTTP vengono smistate tra i vari presenter dell'applicazione, secondo il pattern Model-View-Presenter, ovvero:
\begin{enumerate}
	\item le routes inviano le richieste dell'utente che vengono realizzate da uno specifico presenter;
	\item i presenters operano sul modello, ottengono i dati necessari alla visualizzazione delle pagine e le costruiscono;
	\item le views sono passive e definiscono come sono costituite le varie pagine che verranno visualizzate dall'utente.
\end{enumerate}

\subparagraph*{Routes di visualizzazione}
\begin{itemize}
	\item \texttt{/home}: visualizza la vista principale dell'applicazione, con una barra di input di testo per lo svolgimento di un esercizio;
	\item \texttt{/profile}: visualizza il profilo dell'utente autenticato;
	\item \texttt{/exercise}: visualizza un form per lo svolgimento di un esercizio;
	\item \texttt{/exercise/insert}: visualizza un form per l'inserimento di un esercizio;
	\item \texttt{/registration}: visualizza un form per la registrazione alla piattaforma;
	\item \texttt{/class}: visualizza l'area di gestione di una singola classe;
	\item \texttt{/developer}: visualizza l'area principale dello sviluppatore;
	\item \texttt{/classes}: visualizza l'area di gestione delle classi;
	\item \texttt{/exercises}: visualizza l'area di gestione degli esercizi e delle relative soluzioni;
	\item \texttt{/exercise/search}: visualizza la pagina di ricerca degli esercizi;
	\item \texttt{/student/insert}: visualizza la pagina di ricerca e inserimento degli studenti;
	\item \texttt{/class/exercise/search}: visualizza la pagina di ricerca e di inserimento di un esercizio a una classe.
\end{itemize}

\subparagraph*{Routes di utilità}
\begin{itemize}
	\item \texttt{/checklogin}: permette di controllare l'identità di un utente che ha richiesto l'autenticazione;
	\item \texttt{/saveuser}: permette la scrittura delle credenziali di un utente nel database;
	\item \texttt{exercise/save}: permette la scrittura dei dati di un esercizio nel database;
	\item \texttt{/deletestudent}: permette l'eliminazione di uno studente da una classe;
	\item \texttt{/deleteexercise}: permette l'eliminazione di uno esercizio tra quelli assegnati ad una classe;
	\item \texttt{/addstudent}: permette di aggiungere uno studente a una classe;
	\item \texttt{/addexercise}: permette di assegnare un esercizio a una classe;
	\item \texttt{/checkdeveloper}: controlla la password dello sviluppatore;
	\item \texttt{/download}: permette di scaricare il modello;
	\item \texttt{/exercise/update}: permette di aggiornare la soluzione di un esercizio;
	\item \texttt{/insertclass}: permette di inserire una classe;
	\item \texttt{/deleteclass}: permette di eliminare una classe;
	\item \texttt{/update}: permette la modifica dei dati del profilo di un utente;
	\item \texttt{/logout}: permette di effettuare il logout;
	\item \texttt{/searchexercise}: permette la ricerca di un esercizio;
	\item \texttt{/searchstudent}: permette la ricerca di uno studente.
	
	
\end{itemize}