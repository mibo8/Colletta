\subsection{Scopo del capitolo}
Questo capitolo è rivolto a chiunque desideri contribuire allo sviluppo dell'applicazione mantenendeo lo stesso ambiente di sviluppo utilizzato dai creatori del progetto.
L'obiettivo di questo capitolo è quindi quello di illustrare il procedimento per configurare l'ambiente di sviluppo in modo tale che possa essere esattamente equivalente a quello utilizzato dal gruppo Ottobit per lo sviluppo di Colletta.

\subsection{WebStorm}
Webstorm è un IDE di JetBrains per lo sviluppo di applicazioni di Web application, che supporta tutti i linguaggi scelti per lo sviluppo dell'applicazione Colletta, ovvero Javascript, Node.js, HTML e CSS.
L'editor offre varie funzionalità, tra cui:
\begin{itemize}
	\item Node.js Debugger
	\item TSLint
\end{itemize}
È disponibile sia per Windows, che per Linux come che per MacOS.
Può essere scaricato ed installato gratuitamente registrandosi nel \footnote{\url{https://www.jetbrains.com/webstorm/}}sito ufficiale utilizzando un e-mail universitaria.

\subsection{TSLint}
Al fine di evitare errori di stile del codice, in WebStorm dovrà essere abilitato TSLint, nel seguente modo: Settings | Languages \& Frameworks | TypeScript | TSLint.

Utilizzate il seguente comando per installare TSLint:
\begin{center}
		\begin{minipage}{0.5\textwidth}
			\begin{lstlisting}[caption=Installazione per l'analisi statica,numbers=none]			
		$ npm install tslint typescript
			\end{lstlisting}
		\end{minipage}
\end{center}
Per configurare correttamente TSLint è necessario impostare il percorso di installazione dello stesso, nell'area di Webstorm sovracitata.
Inoltre per fare in modo che TSLint funzioni correttamente è necessario configurarlo, specificando nell'impostazione ``Configutation file'' il file tsconfig.json, presente all'interno della cartella principale del sorgente dell'applicazione. È possibile quindi settare l'impostazione ``Configutation file'' su Automated search o specificare manualmente il percorso del file appena citato.

\subsection{TypeDoc}
Per la generazione automatica della documentazione in stile JavaDoc, è stato utilizzato il software TypeDoc. 
Per essere utilizzato TypeDoc necessita di essere installato, tramite npm, nel vostro sistema. Al fine di effettuare tale operazione utilizzate il seguente comando:

\begin{center}
		\begin{minipage}{0.5\textwidth}
		\begin{lstlisting}[caption=Installazione di TypeDoc per la generazione della documentazione,numbers=none]
		$ npm install --global typedoc
			\end{lstlisting}		
		\end{minipage}
\end{center}

Ricordiamo al lettore che il codice dovrà essere commentato in modo esaustivo e conciso, non dovranno essere una descrizione della sintassi del linguaggio di programmazione in uso. Inoltre i commenti dovranno essere in inglese. Di seguito riportiamo anche un esempio:

\begin{lstlisting}[caption=Esempio di commento ad un metodo]
/**
* This method writes a sentence in the database.
* @param sentence - the sentence to write
* @returns {number} returns the key of the sentence written
*/
writeSentence(sentence){
...
}
\end{lstlisting}

Per utilizzare TypeDoc e produrre la documentazione automatica basata sui commenti introdotti nel sorgente prodotto è necessario invocare il seguente comando dalla cartella principale dell'applicazione.

\begin{center}
	\begin{minipage}{0.5\textwidth}
		\begin{lstlisting}[caption=Comandi per la generazione della documentazione,numbers=none]
			$ typedoc --out docs ./src
		\end{lstlisting}
	\end{minipage}
\end{center}
	
\subsection{Test}
All'interno della directory test presente nella cartella principale dell'applicazione. All'interno di essa potrete trovare tutti i sorgenti relativi ai test effettuati sull'applicazione.
Per invocare manualmente la procedura di esecuzione dei test è necessario intallare le librerie Mocha\footnote{\url{https://www.npmjs.com/package/mocha}} e Chai\footnote{\url{https://www.npmjs.com/package/chai}}, utilizzando i seguenti comandi:

\begin{center}
	\begin{minipage}{0.5\textwidth}
		\begin{lstlisting}[caption=Comandi per l'esecuzione dei test,numbers=none]
			$ npm i --global mocha
			$ npm install chai
			$ npm install --global nyc
		\end{lstlisting}
	\end{minipage}
\end{center}
\noindent
Una volta implementato il codice dei test desiderati questi verranno automaticamente eseguiti dalla deployment pipeline, ma se desiderate eseguirli manualmente potrete utilizzare il seguente comando:
\begin{center}
	\begin{minipage}{0.5\textwidth}
		\begin{lstlisting}[caption=Comandi per l'esecuzione dei test,numbers=none]
			$ npm test
		\end{lstlisting}
	\end{minipage}
\end{center}

\newpage

