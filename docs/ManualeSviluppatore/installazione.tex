\subsection{Requisiti software}
Colletta è in generale compatibile con qualsiasi dispositivo, tuttavia garantiamo il corretto funzionamento per i seguenti software.
\begin{itemize}
	\item Node.js versione 8.15
	\item npm versione 6.4.1
	\item Browser supportati:
	\begin{itemize}
		\item Google Chrome 73
		\item Mozilla Firefox 61
		\item Safari 12
	\end{itemize}
	\item Sistemi operativi supportati:
	\begin{itemize}
		\item Linux Ubuntu
		\item MacOS Mojave
	\end{itemize}
\end{itemize}
\subsection{Requisiti hardware}
\subsubsection{Requisiti per Linux}
\begin{itemize}
	\item Sistema: Ubuntu 17.10
	\item RAM: 512MB
	\item Hard-drive: 1GB di spazio
	\item Connessione ad Internet: obbligatoria
\end{itemize}
\subsubsection{Requisiti per Mac}
\begin{itemize}
	\item Sistema: MacOS Mojave 10.14.4
	\item RAM: 512MB
	\item Hard-drive: 1GB di spazio
	\item Connessione ad Internet: obbligatoria
\end{itemize}
\subsection{Come installare l'applicazione}
Per il momento l'applicazione necessita di essere installata ed eseguita in locale.
\begin{itemize}
	\item \textbf{Download dell'applicazione:}\\
Recatevi al seguente indirizzo e scaricate la versione corrispondente al vostro sistema operativo.
\begin{center}
	\url{https://github.com/ottoBitPd/colletta} 	
\end{center}


	\item \textbf{Installazione di Node.js:}\\
Scaricate Node.js versione 8.15 o superiore dal sito ufficiale di seguito riportato 
\begin{center}
	\url{https://nodejs.org/en/download/}
\end{center}

	\item \textbf{Download e installazione di tutte le librarie necessarie:}\\ 
Recatevi all' interno della sottocartella /code contenuta nella cartella principale dell'applicazione precedentemente scaricata. Utilizzando una shell eseguite il seguente comando per scaricare ed installare le librerie necessarie all'esecuzione dell'applicazione in locale.

\begin{center}
	\begin{minipage}{0.5\textwidth}
		\begin{lstlisting}[caption=Installazione per l'analisi statica,numbers=none]
			$ npm install
		\end{lstlisting}
	\end{minipage}
\end{center}

	\item \textbf{Eseguire il server in locale:}\\
Aprite una finestra di comando nela cartella principale dell'applicazione e usate i seguenti comandi per eseguire il server in locale.

\begin{center}
	\begin{minipage}{0.5\textwidth}
		\begin{lstlisting}[caption=Avvio della piattaforma,numbers=none]	
			$ node public/index.js
		\end{lstlisting}
	\end{minipage}
\end{center}

\end{itemize}
