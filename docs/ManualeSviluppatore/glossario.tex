\subsection*{A}
	\subsubsection*{Apprendimento automatico}
	L'apprendimento automatico rappresenta un insieme di meccanismi che permettono ad un calcolatore
	di migliorare nel tempo l'esecuzione di un esercizio.
	
	\subsection*{C}
	\subsubsection*{CRUD}
	Le quattro funzioni basilari di un servizio che garantisce la persistenza dei dati, come i database.
	CRUD sono le iniziali di Create,Read,Update,Delete.
	
	\subsection*{D}
	\subsubsection*{Database}
	Archivio di dati strutturato in modo da razionalizzare la gestione e l'aggiornamento delle informazioni e da permettere lo svolgimento di ricerche complesse.
	
	\subsection*{E}
	\subsubsection*{Template Engine}
	Un Template Engine è un insieme di librerie e/o codici che permette di separare il contenuto (codice, dati ecc.) dalla presentazione (layout, html, css).
	
	\subsection*{F}
	\subsubsection*{Framework}
	Struttura di supporto per l'organizzazione e progettazione di prodotti software. Include software di supporto, librerie, un linguaggio per gli script e altri software che possono aiutare a mettere insieme le varie componenti di un progetto.
	
	\subsection*{J}
	\subsubsection*{Javascript}
	E' un linguaggio di scripting orientato agli oggetti e agli eventi, comunemente utilizzato nella programmazione Web lato client per la creazione, in siti web e applicazioni web, di effetti dinamici interattivi tramite funzioni di script invocate da eventi innescati a loro volta in vari modi dall'utente sulla pagina web in uso (mouse, tastiera, caricamento della pagina..).
	
		\subsection*{N}
	\subsubsection*{NodeJS}
	Node.js è una runtime di JavaScript Open source multipiattaforma orientato agli eventi per l'esecuzione di codice JavaScript Server-side.
	
	\subsection*{P}
	\subsubsection*{Pos-Tagging}
	E' l'associare ad una parola un tag che ne descriva la propria classe grammaticale.
	
	\subsection*{R}
	\subsubsection*{Routing}
	Modo di determinare come un’applicazione risponde a una richiesta client a un endpoint particolare,
	il quale è un URI (o percorso) e un metodo di richiesta HTTP specifico (GET, POST e così via).
	Ciascuna route può disporre di una o più funzioni di un handler, le quali vengono eseguite quando si
	trova una corrispondenza per la route.
	
	\subsection*{S}
	\subsubsection*{Scalabilità}
	La caratteristica di un sistema software o hardware facilmente modificabile nel caso di variazioni
	notevoli della mole o della tipologia dei dati trattati.
	
	\subsection*{T}
	\subsubsection*{Throughput}
			Quantità di dati trasmessi in un’unità di tempo, ovvero la capacità di trasmissione “effettiva”.
			
	\subsubsection*{TSLint}
	E' uno strumento di analisi statica estensibile che controlla il codice TypeScript per la leggibilità, manutenibilità e gli errori di funzionalità.
	
	\subsection*{W}
	\subsubsection*{Web Browser}
	E' una applicazione per l'accesso alle risorse del web.

	
	
	