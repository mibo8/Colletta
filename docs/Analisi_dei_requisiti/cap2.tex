\subsection{Obiettivi del prodotto}
	L'obiettivo del prodotto è quello di fornire agli utenti una piattaforma online in cui sia possibile trovare, creare e svolgere esercizi di analisi grammaticale. 
Con lo scopo di poter formare un correttore automatico degli esercizi, i dati verranno raccolti e immagazzinati. 
La proponente ha imposto i seguenti vincoli sull'implementazione del prodotto:

\subsubsection{Obbligatori}
\begin{itemize}
	\item l'utente deve poter chiedere al sistema di svolgere automaticamente un esercizio mediante un software basato sull'apprendimento supervisionato;
	\item l'utente deve poter correggere l'output automatico generato dal correttore e di salvare il risultato finale;
	\item l'utente deve poter scaricare i dati collezionati nella piattaforma.
\end{itemize}
\subsubsection{Desiderabili}
\begin{itemize}
	\item l'utente dovrebbe avere la possibilità di svolgere e correggere gli esercizi in più lingue;
	\item l'utente dovrebbe avere la possibilità di personalizzare i dati prima di scaricarli dalla piattaforma;
	\item l'utente dovrebbe avere la possibilità di differenziare tra dati pubblici e privati.
\end{itemize}
\subsubsection{Opzionali}
\begin{itemize}
	\item l'utente potrebbe voler salvare l'intera cronologia delle modifiche di alcuni dati;
	\item l'utente potrebbe voler creare e/o scaricare un modello direttamente dalla piattaforma.
\end{itemize}

\subsection{Tipologie di utenti}
L'applicazione richiesta prevede l'individuazione di tre categorie di utenti, differenti per obiettivi:
\begin{itemize}
	\item \textbf{Insegnante:} gli insegnanti hanno interesse a preparare in modo semplice e rapido degli esercizi per i propri allievi;
	\item \textbf{Allievi:} gli allievi hanno interesse a svolgere degli esercizi ed avere nell'immediato la correzione;
	\item \textbf{Sviluppatori:} gli sviluppatori hanno interesse a raccogliere dati dalla piattaforma per migliorarne l'automatizzazione.
\end{itemize}
