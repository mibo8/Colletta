\subsection{Obiettivi del prodotto}
	L'obiettivo del prodotto è quello di fornire agli utenti una piattaforma online in cui sia possibile trovare, creare e svolgere esercizi di analisi grammaticale. 
Con lo scopo di poter formare un correttore automatico degli esercizi, i dati verranno raccolti e immagazzinati. 
La proponente ha imposto i seguenti vincoli sull'implementazione del prodotto:

\subsubsection{Obbligatori}
\begin{itemize}
	\item l'utente deve poter chiedere al sistema di svolgere automaticamente un esercizio mediante un software basato sull'apprendimento supervisionato;
	\item l'utente deve poter correggere l'output automatico generato dal correttore e di salvare il risultato finale;
	\item l'utente deve poter scaricare i dati collezionati nella piattaforma.
\end{itemize}
\subsubsection{Desiderabili}
	\item l'utente dovrebbe avere la possibilità di svolgere e correggere gli esercizi in più lingue;
	\item l'utente dovrebbe avere la possibilità di personalizzare i dati prima di scaricarli dalla piattaforma;
	\item l'utente dovrebbe avere la possibilità di differenziare tra dati pubblici e privati.
\subsubsection{Opzionali}
	\item l'utente potrebbe voler salvare l'intera cronologia delle modifiche di alcuni dati;
	\item l'utente potrebbe voler creare e/o scaricare un modello direttamente dalla piattaforma.
	
\subsection{Funzioni del prodotto}


\subsection{Tipologie di utenti}
L'applicazione richiesta prevede l'individuazione di tre categorie di utenti, differenti per obiettivi:
\begin{itemize}
	\item \textbf{Insegnante:} gli insegnanti hanno interesse a preparare degli esercizi per i loro studenti ed assegnarglieli;
	\item \textbf{Allievi:} gli allievi hanno interesse a svolgere degli esercizi ed avere nell'immediato la correzione;
	\item \textbf{Sviluppatori:} gli sviluppatori hanno interesse a raccogliere dati dalla piattaforma per migliorarne l'automatizzazione.
\end{itemize}

\subsection{Piattaforme di esecuzione}
	La proponente lascia libera scelta al gruppo sulla piattaforma di esecuzione del prodotto. 
La scelta è ricaduta sulla piattaforma web -piuttosto che su quella mobile- pensando alla tipologia di utenti che potrebbe usare il sistema: considerando la tipologia di esercizio, il gruppo ha ipotizzato che la maggior parte degli utenti di tipo "allievo" dovrebbero essere studenti delle scuole primarie di primo (scuole elementari) e secondo grado (scuole medie) e quindi, non per forza in possesso di uno smartphone su cui scaricare ed utilizzare un'app.
