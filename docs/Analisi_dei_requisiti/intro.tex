\subsection{Scopo del documento}
	Il presente documento ha lo scopo di raccogliere i casi d'uso$^*$ ed i requisiti individuati dal gruppo durante la fase di analisi e dedotti dalla documentazione fornita riguardo il capitolato d'appalto proposto dall'azienda MIVOQ (C2). La formulazione di questi segue le regole esposte nel documento \textit{NormeDiProgetto\_v2.0.0}.

\subsection{Scopo del prodotto}
%copiata dal PdP vedere se cosa cambiare
	Il prodotto richiesto dalla proponente è una piattaforma che permetta la raccolta di dati in modo implicito tramite la risoluzione di esercizi. Tali dati devono essere utilizzati per addestrare un software di apprendimento automatico$^*$ già esistente che, a sua volta, deve essere in grado di fornire una soluzione agli esercizi proposti. L'obiettivo del prodotto potrà essere raggiunto tramite l'impiego di un database$^*$ che garantisca la permanenza dei dati, il software di apprendimento automatico e un'interfaccia (web o di un'applicazione mobile) che permetta l'interazione con gli utenti.

\subsection{Glossario}
	All'interno del documento è possibile trovare termini ambigui: in tal caso, tali termini possono essere trovati nel Glossario insieme alla relativa spiegazione. I termini del glossario vengono indicati con un * in apice.
	
\subsection{Riferimenti}
	\subsubsection{Normativi}
	\begin{itemize}
		\item \textit{NormeDiProgetto\_v2.0.0}
		\item Documentazione capitolato 2 \footnote{\url{https://www.math.unipd.it/~tullio/IS-1/2018/Progetto/C2.pdf}}
	
	\end{itemize}
	\subsubsection{Informativi}
	\begin{itemize}
		\item Slide del corso di Ingegneria del software A.A.18/19 - Analisi dei requisiti \footnote{\url{https://www.math.unipd.it/~tullio/IS-1/2018/Dispense/L08.pdf}}
		\item Slide del corso di Ingegneria del software A.A.18/19 - Diagrammi delle classi \footnote{\url{https://www.math.unipd.it/~tullio/IS-1/2018/Dispense/E03b.pdf}}
	\end{itemize}
		