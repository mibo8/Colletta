In questa sezione viene assegnato un codice ad ogni requisito al fine di facilitarne l'identificazione. L'assegnazione avverrà secondo le condizioni stabilite dal gruppo nel documento \textit{NormeDiProgetto\_v2.0.0}, riportate in seguito. 
\subsection{Denominazione dei requisiti}
Si vuole associare un identificatore univoco per ogni requisito individuato durante l'analisi dei requisiti. Si \`e deciso di utilizzare il seguente formato:
  \begin{center}
    R[Priorità][Tipo][Codice]
	\end{center}
dove:
	\begin{itemize}
	\item il campo “\textbf{Priorità}” può assumere uno dei seguenti 	valori:
		\begin{itemize}
  		\item \textbf{O}: indica un requisito obbligatorio, irrinunciabile per il committente;
		\item \textbf{D}: indica un requisito desiderabile ma non strettamente necessario;
		\item \textbf{P}: indica un requisito opzionale, che potrebbe venir soddisfatto o meno senza che il prodotto risulti mancante di funzionalità essenziali.
		\end{itemize}
	\item il campo “\textbf{Tipo}” può assumere uno dei seguenti valori:
		\begin{itemize}
  		\item \textbf{F}: definisce un requisito funzionale, ovvero un requisito che indica quale deve essere la reazione del software in specifici casi (ad esempio con  un determinato input);
		\item \textbf{Q}: definisce un requisito di qualità$^*$, ovvero un requisito votato a garantire efficienza, efficacia e qualità al prodotto;
		\item \textbf{V}: definisce un requisito di vincolo, ovvero un requisito imposto dalla proponente del capitolato;
		\item \textbf{R}: definisce un requisito prestazionale, ovvero un requisito relativo alle prestazioni di sistema.
		\end{itemize}
	\item il campo “\textbf{Codice}” assumerà un valore numerico intero positivo, univoco ed incrementale.
	\end{itemize}
\newpage
\subsection{Requisiti funzionali}
\begin{tabularx}{\textwidth}{| c | p{10cm} | X |}
		\rowcolor{LightBlue}
		\color{white}\bfseries Requisito & \color{white}\bfseries Descrizione & \color{white}\bfseries Fonti\\[0.25cm]
		ROF1 & L'utente non riconosciuto si registra alla piattaforma. & UC-1 \newline Interno\\
		ROF24 & L'utente non riconosciuto inserisce il proprio nome mentre si registra alla piattaforma. & UC-1.1 \newline Interno\\
		ROF25 & L'utente non riconosciuto inserisce il proprio cognome mentre si registra alla piattaforma . & UC-1.2 \newline Interno\\
		ROF26 & L'utente non riconosciuto inserisce l'username mentre si registra alla piattaforma. & UC-1.3 \newline Interno\\
		ROF27 & L'utente non riconosciuto inserisce la propria email mentre si registra alla piattaforma. & UC-1.4 \newline Interno\\
		ROF28 & L'utente non riconosciuto inserisce la password mentre si registra alla piattaforma. & UC-1.5 \newline Interno\\
		ROF29 & L'utente non riconosciuto inserisce il nome della scuola che frequenta mentre si registra alla piattaforma. & UC-1.6 \newline Interno\\
		ROF30 & L'utente non riconosciuto inserisce la città a cui la scuola appartiene mentre si registra alla piattaforma. & UC-1.7 \newline Interno\\
		ROF22 & L'utente non riconosciuto si registra alla piattaforma come insegnante. & UC-33 \newline Interno\\
		ROF31 & L'utente non riconosciuto inserisce il proprio codice INPS mentre si registra alla piattaforma come insegnante. & UC-33.1 \newline Interno\\
		ROF32 & L'utente non riconosciuto si registra alla piattaforma come allievo. & UC-61 \newline Interno\\	
		ROF2 & L'utente non riconosciuto esegue l'accesso alla piattaforma utilizzando le sue credenziali. & UC-2 \newline Interno\\
		ROF33 & L'utente non riconosciuto inserisce il proprio username mentre esegue l'accesso alla piattaforma. & UC-2.1 \newline Interno\\
		ROF34 & L'utente non riconosciuto inserisce la propria password mentre esegue l'accesso alla piattaforma. & UC-2.2 \newline Interno\\
		RDF1 & L'utente riconosciuto modifica i dati del proprio profilo personale. & UC-3 \newline Interno\\
		RDF26 & L'utente riconosciuto modifica il proprio username mentre modifica il profilo. & UC-3.1 \newline Interno\\
		RDF27 & L'utente riconosciuto modifica la propria password mentre modifica il profilo. & UC-3.2 \newline Interno\\
		RDF28 & L'utente riconosciuto modifica il nome della scuola che frequenta mentre modifica il profilo. & UC-3.3 \newline Interno\\
		RDF29 & L'utente riconosciuto modifica la città a cui la scuola appartiene mentre modifica il profilo. & UC-3.4 \newline Interno\\
		RDF2 & Il moderatore verifica le credenziali di un utente che richiede la registrazione come insegnante. & UC-4 \newline Interno\\
		RDF3 & L'utente non riconosciuto viene avvisato in caso di errore nell'inserimento dei dati al momento della registrazione. & UC-5 \newline Interno\\
		ROF3 & L'utente deve poter ricercare degli esercizi sulla piattaforma. & UC-6 \newline Capitolato\\
		RPF1 & Durante la ricerca, l'utente ha la possibilità di impostare dei filtri per raffinarla. & UC-6.1 \newline Interno\\
		RPF2 & Durante la ricerca, l'utente può impostare il filtro secondo gli autori. & UC-6.1.1 \newline Interno\\
		RPF3 & Durante la ricerca, l'utente può impostare il filtro secondo la difficoltà. & UC-6.1.2 \newline Interno\\
		RPF4 & Durante la ricerca, l'utente può impostare il filtro secondo gli argomenti trattati. & UC-6.1.3 \newline Interno\\
		RDF4 & L'allievo può visualizzare la lista delle classi a cui appartiene. & UC-7 \newline Interno\\
		RPF5 & Il moderatore può eliminare un utente iscritto alla piattaforma. & UC-8 \newline Interno\\
		RPF6 & L'insegnante può modificare una soluzione di un esercizio da lui fornita. & UC-9 \newline Capitolato\\
		RDF5 & L'insegnante, accedendo alla sua area del profilo, può visualizzare la lista degli esercizi da lui creati. & UC-10 \newline Interno\\
		RPF7 & L'insegnante può eliminare una soluzione di un esercizio da lui fornita. & UC-11 \newline Interno\\
		ROF4 & L'insegnante può inserire un esercizio nel sistema. & UC-12 \newline Capitolato\\
		ROF5 & L'insegnante deve inserire la soluzione dell'esercizio che sta creando; può renderla pubblica o privata. & UC-12.1 \newline Capitolato\\
		RPF8 & L'insegnante indica gli argomenti trattati nell'esercizio che sta creando. & UC-12.2 \newline Interno\\
		RPF9 & L'insegnante indica il livello di difficoltà nell'esercizio che sta creando. & UC-12.3 \newline Interno\\
		ROF6 & L'utente deve poter svolgere un esercizio da lui indicato. & UC-13 \newline Capitolato\\
		ROF7 & L'utente deve poter inserire una frase da svolgere o selezionare un esercizio da quelli disponibili sul sistema. & UC-13.1 \newline Capitolato\\
		ROF8 & L'utente deve visualizzare la valutazione dell'esercizio da lui svolto. & UC-13.2 \newline Capitolato\\
		RDF6 & L'allievo, accedendo al proprio profilo, potrà visualizzare i dati relativi ai propri progressi. & UC-14 \newline Capitolato\\
		RDF7 & Lo sviluppatore può ottenere una lista delle annotazioni di una particolare frase. & UC-15 \newline Capitolato\\
		RDF8 & Lo sviluppatore deve poter filtrare i dati trovati durante la ricerca ottenendo una lista di annotazioni. & UC-15.1 \newline Capitolato\\
		RDF9 & Lo sviluppatore può impostare un filtro temporale per la ricerca delle annotazioni. & UC-15.1.1 \newline Interno\\
		RDF10 & Lo sviluppatore deve poter includere dalla ricerca di annotazioni solo uno o più utenti. & UC-15.1.2 \newline Capitolato\\
		RDF18 & Lo sviluppatore durante l'inclusione di un utente deve poter cercare degli utenti & UC-15.1.2.1 \newline Interno\\		
		RPF10 & Lo sviluppatore deve poter visualizzare i dati relativi ad una particolare annotazione. & UC-16 \newline Capitolato\\
		RPF11 & Lo sviluppatore deve poter visualizzare lo storico delle annotazioni. & UC-17 \newline Capitolato\\
		RPF12 & Lo sviluppatore può ordinare la lista dei risultati ottenuti dalla ricerca tramite determinati parametri. & UC-18 \newline Interno\\	
		ROF9 & Lo sviluppatore deve poter scaricare un file contenente i dati relativi agli esercizi ottenuti con la ricerca. & UC-19 \newline Capitolato\\
		RPF13 & Lo sviluppatore può visualizzare le informazioni relative ad un dataset. & UC-20 \newline Interno\\
		RPF14 & Lo sviluppatore deve poter scaricare un modello. & UC-21 \newline Capitolato\\
		RPF15 & Lo sviluppatore deve poter creare un modello tramite la piattaforma. & UC-22 \newline Capitolato\\ 
		RPF16 & Il moderatore deve poter eliminare uno qualsiasi degli esercizi inseriti nel sistema. & UC-23 \newline Interno\\
		ROF10 & L'insegnante deve poter creare una nuova classe. & UC-24 \newline Interno\\
		ROF11 & L'insegnante deve poter eliminare una classe dal sistema. & UC-25 \newline Interno\\
		ROF12 & L'insegnante deve poter aggiungere degli alunni ad una classe. & UC-26 \newline Interno\\
		RPF17 & L'insegnante deve indicare gli alunni da inserire nella classe. & UC-26.1 \newline Interno\\
		RPF18 & L'insegnante può ricercare un alunno da inserire nella classe tramite username. & UC-26.1.1 \newline Interno\\
		ROF13 & L'insegnante deve poter aggiungere degli esercizi a quelli assegnati ad una classe. & UC-27 \newline Interno\\
		RPF19 & L'insegnante potrebbe voler visualizzare i progressi degli alunni di una propria classe. & UC-28 \newline Interno\\
		ROF14 & L'insegnante deve poter eliminare un alunno dalla lista di quelli iscritti ad una delle proprie classi. & UC-29 \newline Interno\\
		ROF15 & L'insegnante deve poter visualizzare la lista degli alunni iscritti ad una delle sue classi. & UC-30 \newline Interno\\
		ROF16 & L'insegnante deve poter visualizzare la lista delle proprie classi. & UC-31 \newline Interno\\
		RDF11 & L'allievo può annullare l'iscrizione ad una classe. & UC-32 \newline Interno\\
		ROF17 & L'utente non riconosciuto viene avvisato in caso di errore nell'inserimento dei dati al momento dell'autenticazione. & UC-34 \newline Interno\\
		ROF18 & L'utente riconosciuto viene avvisato in caso di errore nella modifica delle proprie informazioni. & UC-35 \newline Interno\\
		RPF20 & L'utente non riconosciuto deve ricevere una e-mail per la conferma di iscrizione come insegnante. & UC-36 \newline Interno\\
		ROF19 & L'insegnante visualizzerà un messaggio di errore nel caso in cui stia inserendo una frase vuota come esercizio. & UC-37 \newline Interno\\
		RDF24 & L'utente potrà accedere alla pagina di registrazione alla piattaforma. & UC-38 \newline Interno\\
		ROF20 & L'utente può aggiungere una frase da svolgere come esercizio. & UC-39 \newline Interno\\
		ROF21 & L'utente deve poter selezionare un esercizio da svolgere tra quelli ricercati nella piattaforma. & UC-40 \newline Interno\\
		RDF12 & Lo sviluppatore visualizzerà un messaggio di errore nel caso inserisca una data non valida durante il filtraggio di annotazioni in base temporale. & UC-41 \newline Interno\\
		RDF13 & Lo sviluppatore visualizzerà un messaggio di errore nel caso in cui indichi un path non esistente al momento del download di file. & UC-42 \newline Interno\\
		RDF14 & Lo sviluppatore visualizzerà un messaggio di errore se inserirà un dataset in un formato errato. & UC-43 \newline Interno\\
		RPF27 & Lo sviluppatore può cambiare modello utilizzato dal software di apprendimento automatico. & UC-44 \newline Interno\\
		RDF16 & L'allievo può visualizzare le informazioni riguardanti una classe a cui appartiene e gli esercizi assegnati. & UC-45 \newline Interno\\
		RPF21 & L'utente può segnalare un esercizio per abuso delle regole comportamentali. & UC-46 \newline Interno\\
		RPF22 & Il moderatore può visualizzare la lista delle segnalazioni effettuate dagli utenti. & UC-47 \newline Interno\\
		RPF23 & Il moderatore può eliminare una segnalazione dalla relativa lista. & UC-48 \newline Interno\\
		RDF17 & L'insegnante può visualizzare l'area per la gestione delle classi. & UC-49 \newline Interno\\
		RDF19 & Il moderatore può visualizzare la lista degli esercizi inseriti nella piattaforma. & UC-50 \newline Interno\\
		RPF24 & Il moderatore può ricercare gli esercizi inserendo la frase o una parte di essa. & UC-51 \newline Interno\\
		RDF20 & Il moderatore può visualizzare la lista degli utenti iscritti alla piattaforma. & UC-52 \newline Interno\\
		RPF25 & Il moderatore può ricercare gli utenti inserendo l'username o una parte di esso. & UC-53 \newline Interno\\
		RDF21 & L'insegnante può ricercare gli esercizi che ha inserito inserendo la frase o una parte di essa. & UC-54 \newline Interno\\
		RDF22 & L'utente riconosciuto può accedere alla vista del proprio profilo personale. & UC-55 \newline Interno\\
		RDF23 & Lo sviluppatore può visualizzare la lista dei modelli disponibili. & UC-56 \newline Interno\\
		RPF26 & Lo sviluppatore può visualizzare informazioni riguardanti i modelli disponibili nella piattaforma. & UC-57 \newline Interno\\
		RDF25 & L'utente riconosciuto può disconnettersi dalla piattaforma. & UC-58 \newline Interno\\
		ROF23 & L'insegnante deve poter accedere all'area di inserimento nuovo esercizio. & UC-59 \newline Interno\\
		RPF28 & L'allievo può selezionare un esercizio assegnato. & UC-60 \newline Interno\\
		\hline
		\caption{Tabella dei requisiti funzionali}
\end{tabularx}

\subsection{Requisiti di vincolo}
\begin{longtable}{| c | p{10cm} | c |}
		\rowcolor{LightBlue}
		\color{white}\bfseries Requisito & \color{white}\bfseries Descrizione & \color{white}\bfseries Fonti\\[0.25cm]
		ROV1 & Utilizzo del software di apprendimento automatico per il pos-tagging$^*$ Hunpos per lo svolgimento degli esercizi. & Capitolato \\
		RDV1 & Utilizzo di Google Firebase per l'immagazzinamento dei dati. & Capitolato \\
		ROV2 & I dati dovranno essere immagazzinati opportunamente e dovranno essere esportabili in uno o più formati. & Capitolato \\
		RPV1 & Dovrebbe essere consentita la raccolta di dati in più lingue. & Capitolato \\ 
		ROV3 & L’interfaccia web deve essere sviluppata utilizzando HTML5, CSS3 e JavaScript & Interno\\
		ROV4 & L'implementazione della gestione del database deve essere sviluppata utilizzando JavaScript(Node.js) & Interno\\
		ROV5 & La parte backend del Softwere deve essere sviluppata utilizzando JavaScript(Node.js) & Interno\\
		\hline
		\caption{Tabella dei requisiti di vincolo}
\end{longtable}

\subsection{Requisiti di qualità}
\begin{longtable}{| c | p{10cm} | c |}
		\rowcolor{LightBlue}
		\color{white}\bfseries Requisito & \color{white}\bfseries Descrizione & \color{white}\bfseries Fonti\\[0.25cm]
		ROQ1 & Il gruppo deve fornire un'analisi dei requisiti come documentazione dell'applicazione alla proponente. & Capitolato \\
		ROQ2 & Il gruppo deve fornire una descrizione tecnica del prodotto alla proponente. & Capitolato \\ 
		RDQ1 & I commenti, i nomi delle funzioni e delle variabili del codice dovrebbero essere esplicativi ed in lingua inglese. & Capitolato \\ 
		RDQ2 & Il numero di parametri delle procedure deve essere al massimo 3. & Capitolato \\
		RPQ1 & La piattaforma deve essere fornita di un codice di comportamento per gli utenti & Interno\\
		RDQ3 & I documenti devono avere un indice di \textit{Gulpease} di almeno 50 & Interno\\
		ROQ3 & Tutte le norme stabilite nel documento \textit{NormeDiProgetto\_v2.0.0} devono essere rispettate & Interno\\
		RDQ4 & Il numero di linee di codice di ogni procedura non deve essere maggiore di 20. & Capitolato\\
		RDQ5 & Il valore della complessità ciclomatica$^*$ (stabilito nelle \textit{NormeDiProgetto\_v2.0.0}) deve essere minore di 10. & Capitolato\\
		\hline
		\caption{Tabella dei requisiti di qualità}
\end{longtable}