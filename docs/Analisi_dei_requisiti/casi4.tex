\subsection{Attore Allievo}
	\subsubsection{UC-1 Ricerca esercizi disponibili}
		\begin{itemize}
			\item Attori: Allievo;
			\item Precondizione: l'allievo si trova nella vista principale dell'applicazione;
			\item Postcondizione: l'allievo ottiene una lista degli esercizi filtrati.
			\item Scenario principale:
				\begin{enumerate}
					\item l'allievo accede all'area dedicata alla ricerca degli esercizi;
					\item l'allievo seleziona i filtri.
				\end{enumerate}
			\item Estensioni:
				\begin{itemize}
					\item UC-1.1 Nel caso in cui nessun risultato sia stato trovato;
					\item UC-2 Nel caso in cui si voglia aggiungere un esercizio;
					\item UC-3 Nel caso in cui si voglia eseguire un esercizio.
				\end{itemize}
		\end{itemize}
	\subsubsection{UC-1.1 Nessun risultato trovato}
		\begin{itemize}
			\item Attori: Allievo;
			\item Precondizione: l'allievo si trova nella ricerca degli esercizi;
			\item Postcondizione: l'allievo ottiene un messaggio di errore.
			\item Scenario principale:
				\begin{enumerate}
					\item l'allievo seleziona i filtri;
					\item non sono presenti esercizi disponibili per i filtri selezionati.
				\end{enumerate}
		\end{itemize}
	\subsubsection{UC-2 Aggiunta esercizio}
		\begin{itemize}
			\item Attori: Allievo
			\item Precondizione: l'allievo si trova nella ricerca degli esercizi;
			\item Postcondizione: l'allievo ottiene tra gli esercizi disponibili l'esercizio appena aggiunto.
			\item Scenario principale:
				\begin{enumerate}
					\item l'allievo scrive nella barra una frase.
				\end{enumerate}
			\item Estensioni:
				\begin{itemize}
					\item UC-2.1 Nel caso in cui la frase sia già presente nel database;
					\item UC-2.2 Nel caso in cui la frase non contenga almeno due parole.
				\end{itemize}
		\end{itemize}
	\subsubsection{UC-2.1 Frase già esistente}
		\begin{itemize}
			\item Attori: Allievo
			\item Precondizione: l'allievo si trova nell'aggiunta di un nuovo esercizio;
			\item Postcondizione: l'allievo ottiene un messaggio di errore.
			\item Scenario principale:
				\begin{enumerate}
					\item l'allievo scrive nella barra una frase;
					\item la frase inserita è già presente.
				\end{enumerate}
		\end{itemize}
	\subsubsection{UC-2.2 Frase troppo corta}
		\begin{itemize}
			\item Attori: Allievo
			\item Precondizione: l'allievo si trova nell'aggiunta di un nuovo esercizio;
			\item Postcondizione: l'allievo ottiene un messaggio di errore.
			\item Scenario principale:
				\begin{enumerate}
					\item l'allievo scrive nella barra una frase;
					\item la frase inserita deve contenere almeno due parole.
				\end{enumerate}
		\end{itemize}
	\subsubsection{UC-3 Esecuzione esercizio}
		\begin{itemize}
			\item Attori: Allievo
			\item Precondizione: l'allievo si trova nella ricerca degli esercizi;
			\item Postcondizione: l'allievo dispone dei quesiti dell'esercizio.
			\item Scenario principale:
				\begin{enumerate}
					\item l'allievo sceglie la classe grammaticale dal menu a tendina per ogni parola.
				\end{enumerate}
			\item Estensioni: 
				\begin{itemize}
					\item UC-3.1 Nel caso in cui tutti i campi non fossero completati;
				\end{itemize}
			\item Inclusioni:
				\begin{itemize}
					\item UC-4 Nel caso in cui tutti i campi fossero completati;
				\end{itemize}
			\end{itemize}
	\subsubsection{UC-2.2 Campi non completi}
		\begin{itemize}
			\item Attori: Allievo
			\item Precondizione: l'allievo si trova nell'esecuzione di un esercizio;
			\item Postcondizione: l'allievo ottiene un messaggio di errore.
			\item Scenario principale:
				\begin{enumerate}
					\item l'allievo completa alcuni e non tutti i campi richiesti dall'esercizio;
					\item tutti i campi devono essere completati.
				\end{enumerate}
		\end{itemize}
	\subsubsection{UC-4 Valutazione esercizio}
	\begin{itemize}
			\item Attori: Allievo
			\item Precondizione: L'allievo ha completato l'esecuzione dell'esercizio;
			\item Postcondizione: L'allievo visualizza la valutazione dell'esercizio.
			\item Scenario principale:
				\begin{enumerate}
					\item l'allievo al termine dell'esecuzione dell'esercizio riceve la valutazione.
				\end{enumerate}
			\end{itemize}
	\subsubsection{UC-5 Visualizzazione progressi}
	\begin{itemize}
			\item Attori: Allievo
			\item Precondizione: L'allievo si trova nella vista principale dell'applicazione;
			\item Postcondizione: L'allievo visualizza i progressi svolti fino a quel momento.
			\item Scenario principale:
				\begin{enumerate}
					\item l'allievo prima e/o dopo l'esecuzione di uno o più esercizi può visualizzare i progressi raggiunti.
				\end{enumerate}
			\end{itemize}
	\subsubsection{UC-6 Modifica profilo personale}
	\begin{itemize}
			\item Attori: Allievo
			\item Precondizione: L'allievo si trova nella vista principale dell'applicazione;
			\item Postcondizione: L'allievo ha modificato i dati riguardanti il profilo personale.
			\item Scenario principale:
				\begin{enumerate}
					\item l'allievo modifica i campi di interesse.
				\end{enumerate}
	\end{itemize}