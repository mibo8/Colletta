\section{Gestione delle attività di verifica}
\subsection{Procedura per la risoluzione delle anomalie}
In fase di verifica, i membri che formano il gruppo devono seguire il procedimento di seguito riportato:
\begin{enumerate}
\item Ogni volta che un componente del team risolve una issue* assegnata dal Responsabile di Progetto, la issue stessa passa dallo stato di elaborazione (Doing), allo stato di verifica.
\item Il verificatore analizza i file contenuti nel ramo corrispondente alla issue e li valuta nella loro interezza. In base al tipo di analisi che sta svolgendo, il Verificatore sceglie i procedimenti più appropriati tra quelli descritti in questo documento e stila un resoconto, che può avere risultato positivo o negativo.
Il Verificatore ha il compito di compilare le apposite tabelle in sezione §4, dove segnalerà eventuali anomalie, il risultato delle misurazioni e l'esito. Inoltre le anomalie verranno comunicate tramite un commento nella pagina relativa all'issue in Gitlab. In questo modo, lo strumento invierà automaticamente una notifica al redattore. In caso di esito negativo, la issue torna in stato di Doing. Il Verificatore non ha il compito di fornire eventuali soluzioni per una determinata anomalia.
\item L'autore del file non accettato, deve provvedere a risolvere le anomalie riscontrate dal Verificatore. In caso si necessiti di ulteriori chiarimenti tra Autore e Verificatore, dovranno essere utilizzati i canali di comunicazione che il team si è imposto di usare come dichiarato nelle Norme di Progetto.
\item Il verificatore deve poi assicurarsi che le anomalie siano state corrette e, nel caso in cui non vengano riscontrati ulteriori problemi, accettare il documento.
\item L'approvazione finale spetta al Responsabile di Progetto.
\end{enumerate}


\subsection{Procedure di controllo di qualità di processo}
Come già specificato per verificare la qualità dei processi adotteremo lo standard ISO/IEC 15504. Per far ciò il verificatore dovrà occuparsi di valutare i processi secondo i parametri specificati dallo standard, consultabili nell'appendice A in fondo a questo documento.
Per ogni processo dovranno essere valutati nove attributi assegnando ad ognuno di essi uno dei quattro livelli di misura: 
\begin{itemize}
\item N not implemented, 
\item P partial implemented, 
\item L largely implemented, 
\item F fully implemented.
\end{itemize}
Infine basandosi sui risultati delle valutazioni di questi attributi verrà deciso il grado complessivo di maturazione del processo espresso con uno dei sei livelli offerti dallo standard. I risultati di queste verifiche dovranno essere documentati, per questo il verificatore si occuperà di compilare, con i dati raccolti, le tabelle per la valutazione dei processi previste nella quarta sezione di questo documento.
Durante tutta la fase di controllo della qualità del processo è consigliabile consultare il Piano di qualifica costantemente in particolare le appendici, questo per evitare imprecisioni nella valutazione della qualità sia essa dei processi o del prodotto.