\section{Resoconto attività di verifica}
\subsection{Prodotto}
\subsubsection{Documentazione}
Nella tabella seguente vengono riportati i risultati delle verifiche eseguite sui documenti. Il resoconto contiene le verifiche sia dei documenti esterni, cioè utili al committente, sia interni, utili invece al team Ottobit.\\
{\renewcommand{\arraystretch}{1.5}%
\begin{longtable}{>{\centering\arraybackslash}m{3cm} >{\centering\arraybackslash}m{4cm} >{\centering\arraybackslash}m{5cm} >{\centering\arraybackslash}m{2cm}}
	\rowcolor{LightBlue}
		  \textbf{\textcolor{white}{Data}}
		& \textbf{\textcolor{white}{Autore}}
		& \textbf{\textcolor{white}{Documento}} 
		& \textbf{\textcolor{white}{Versione}}\\
		2019-01-11
		& Gianmarco Pettenuzzo
		& \textit{Analisi dei requisiti}
		& 0.1.0\\
		\rowcolor{LightGray}
		\multicolumn{4}{p{15.25cm}}{\textbf{Descrizione:} 
			Documento corretto e pronto per l'approvazione.
		}\\
		\rowcolor{LightGray}
		\multicolumn{4}{p{15.25cm}}{
			\textbf{Indice di Gulpease:} 82
		}\\
		\rowcolor{LightGray}
		\multicolumn{4}{p{15.25cm}}{
			\textbf{Esito:} Accettato
		}\\
		\hline
		2019-01-11
		& Giovanni Bergo
		& \textit{Piano di qualifica}
		& 0.2.0\\
		\rowcolor{LightGray}
		\multicolumn{4}{p{15.25cm}}{\textbf{Descrizione:} 
			Documento conforme e senza particolari errori da evidenziare.
		}\\
		\rowcolor{LightGray}
		\multicolumn{4}{p{15.25cm}}{
			\textbf{Indice di Gulpease:} 72
		}\\
		\rowcolor{LightGray}
		\multicolumn{4}{p{15.25cm}}{
			\textbf{Esito:} Accettato
		}\\
		\hline
		
		2018-01-11
		& Giovanni Peron
		& \textit{Piano di progetto}
		& 0.4.0\\
		\rowcolor{LightGray}
		\multicolumn{4}{p{15.25cm}}{\textbf{Descrizione:} 
			Nessun problema da segnalare.
		}\\
		\rowcolor{LightGray}
		\multicolumn{4}{p{15.25cm}}{
			\textbf{Indice di Gulpease:} x
		}\\
		\rowcolor{LightGray}
		\multicolumn{4}{p{15.25cm}}{
			\textbf{Esito:} Accettato
		}\\
		\hline
		
		2018-01-10
		& Giovanni Peron
		& \textit{Piano di progetto}
		& 0.3.0\\
		\rowcolor{LightGray}
		\multicolumn{4}{p{15.25cm}}{\textbf{Descrizione:} 
			Nella sezione §4.1.2 sotto il paragrafo Analisi dei requisiti (2018-12-19 - 2018-02-09) da ricontrollare  il corsivo dei riferimenti ai documenti. Nel capitolo §4 posizione delle caption delle tabelle da uniformare. Mancanti le caption nelle ultime sei tabelle del documento.
		}\\
		\rowcolor{LightGray}
		\multicolumn{4}{p{15.25cm}}{
			\textbf{Indice di Gulpease:} 60
		}\\
		\rowcolor{LightGray}
		\multicolumn{4}{p{15.25cm}}{
			\textbf{Esito:} Non accettato
		}\\
		\hline
		
		2018-12-28
		& Giovanni Bergo
		& \textit{Piano di qualifica}
		& 0.1.0\\
		\rowcolor{LightGray}
		\multicolumn{4}{p{15.25cm}}{\textbf{Descrizione:} 
		Footer mancante. Piccoli errori grammaticali e di punteggiatura. Completamento descrizione §1.3 Note esplicative. Url citazioni non collegati al sito web. Introdurre rifermento a sezioni come indicato nelle 	
		textit{Norme di progetto}. Elenco delle tabelle e figure mancanti.
		}\\
		\rowcolor{LightGray}
		\multicolumn{4}{p{15.25cm}}{
			\textbf{Indice di Gulpease:} 66
		}\\
		\rowcolor{LightGray}
		\multicolumn{4}{p{15.25cm}}{
			\textbf{Esito:} Non accettato
		}\\
		\hline
		
		2018-12-26
		& Michele Bortone
		& \textit{Norme di progetto}
		& 0.2.0\\
	\rowcolor{LightGray}
	\multicolumn{4}{p{15.25cm}}{\textbf{Descrizione:} 
	Documento conforme e senza particolari errori da evidenziare.
	Pronto per l'approvazione.
	}\\
	\rowcolor{LightGray}
	\multicolumn{4}{p{15.25cm}}{
	\textbf{Indice di Gulpease:} 81
	}\\
	\rowcolor{LightGray}
	\multicolumn{4}{p{15.25cm}}{
	\textbf{Esito:} Accettato
	}\\
	\hline
		2018-12-18
		& Giovanni Peron
		& \textit{Piano di progetto}
		& 0.2.0\\
		\rowcolor{LightGray}
	\multicolumn{4}{p{15.25cm}}{\textbf{Descrizione:} Nulla da segnalare.
	}\\
	\rowcolor{LightGray}
	\multicolumn{4}{p{15.25cm}}{
	\textbf{Indice di Gulpease:} 86
	}\\
		\rowcolor{LightGray}
	\multicolumn{4}{p{15.25cm}}{
	\textbf{Esito:} Accettato
	}\\
		\hline
				2018-12-16
		& Michele Bortone
		& \textit{Studio di fattibilità}
		& 0.2.0\\
		\rowcolor{LightGray}
	\multicolumn{4}{p{15.25cm}}{\textbf{Descrizione:} 
	Nulla da segnalare.
	}\\
	\rowcolor{LightGray}
	\multicolumn{4}{p{15.25cm}}{
	\textbf{Indice di Gulpease:} 60
	}\\
		\rowcolor{LightGray}
	\multicolumn{4}{p{15.25cm}}{
	\textbf{Esito:} Accettato
	}\\
	\hline
		2018-12-16
		& Giovanni Peron
		& \textit{Piano di progetto}
		& 0.1.0\\
		\rowcolor{LightGray}
	\multicolumn{4}{p{15.25cm}}{\textbf{Descrizione:} Nella tabella del'analisi dei rischi del capitolo §2 ci sono ripetizioni nelle righe R01 e T01 entrambe nella colonna rilevamento. Il contenuto della tabella risulta tagliato a fine pagina 4. In §3.1 a riga 6 suggerisco di inserire Proof of Concept nel glossario. In tutto il documento rivedere il formato delle date secondo le norme di progetto. Per informazioni più dettagliate vedi i commenti scritti nel file relativo al documento.
	}\\
	\rowcolor{LightGray}
	\multicolumn{4}{p{15.25cm}}{
	\textbf{Indice di Gulpease:} 95
	}\\
		\rowcolor{LightGray}
	\multicolumn{4}{p{15.25cm}}{
	\textbf{Esito:} Non accettato
	}\\
		\hline
		2018-12-16
		& Michele Bortone
		& \textit{Norme di progetto}
		& 0.1.0\\
		\rowcolor{LightGray}
	\multicolumn{4}{p{15.25cm}}{\textbf{Descrizione:} 
	Il correttore segnala alcuni errori ortografici.
	Sezione §1 incompleta.
	Sezione §2.1 incompleta.
	Nella sezione §4.1.3.5 è presente un errore nella descrizione delle norme riguardanti l'inserimento delle figure all'interno di un documento. Suggerisco di aggiungere l'obbligo di inserire una breve didascalia dell'immagine corrispondente.
	Nella sezione §5.1.4 è presente un errore riguardo i compiti di ciascun ruolo. Bisogna correggere il redattore dello Studio di fattibilità.
	}\\
	\rowcolor{LightGray}
	\multicolumn{4}{p{15.25cm}}{
	\textbf{Indice di Gulpease:}79
	}\\
		\rowcolor{LightGray}
	\multicolumn{4}{p{15.25cm}}{
	\textbf{Esito:} Non accettato
	}\\
	\hline
		2018-12-16
		& Michele Bortone
		& \textit{Studio di fattibilità}
		& 0.1.0\\
	\rowcolor{LightGray}
	\multicolumn{4}{p{15.25cm}}{\textbf{Descrizione:} 
	Il correttore segnala alcuni errori ortografici.
	Elenchi puntati non conformi alle \textit{Norme di progetto}.
	}\\
	\rowcolor{LightGray}
	\multicolumn{4}{p{15.25cm}}{
	\textbf{Indice di Gulpease:} 58
	}\\
		\rowcolor{LightGray}
	\multicolumn{4}{p{15.25cm}}{
	\textbf{Esito:} Non accettato
	}\\\hline
	
	\caption{Resoconto attività di verifica}
\end{longtable}
}
\subsubsection{Test di sistema}
Di seguito riportiamo una tabella contenente i test di sistema che intendiamo implementare per la verifica dei requisiti funzionali specificati nell'\emph{Analisi dei requisiti}.
{\renewcommand{\arraystretch}{2}%
\begin{longtable}{|>{\centering\arraybackslash}m{1.6cm}|>{\centering\arraybackslash}m{1.7cm}|m{6.41cm}|>{\centering\arraybackslash}m{3.1cm}|}		
	\rowcolor{LightBlue}
		\textbf{\textcolor{white}{Codice\newline test}}
		& \textbf{\textcolor{white}{Codice\newline requisito}}
		& \multicolumn{1}{|c|}{\textbf{\textcolor{white}{ Descrizione}}}
		& \textbf{\textcolor{white}{Stato}}\\

		\hline
		\rowcolor{LightGray}
		TS-RO1
		& ROF1 
		& Verifica che l'utente riesca a registrarsi alla piattaforma creando un account personale. 
		& non implementato\\ \hline
		\rowcolor{white}
		TS-RO2
		& ROF2 
		& Verifica che l'utente possa eseguire l'accesso alla piattaforma utilizzando le sue credenziali.
		& non implementato\\ \hline
		\rowcolor{LightGray}
		TS-RD1
		& RDF1 
		& Verifica che l'utente possa modificare i dati del proprio profilo personale.
		& non implementato\\ \hline
		\rowcolor{white}
		TS-RD2
		& RDF2 
		& Verifica che l'amministratore possa verificare le credenziali di un utente che richiede la registrazione come insegnante. 
		& non implementato\\ \hline
		\rowcolor{LightGray}
		TS-RD3
		& RDF3 
		& Verifica che l'utente venga avvisato in caso di errore nell'inserimento dei dati richiesti.
		& non implementato\\ \hline
		\rowcolor{white}
		TS-RO3		
		& ROF3 
		& Verifica che l'insegnante e l'allievo possano ricercare degli esercizi sulla piattaforma.
		& non implementato\\ \hline
		\rowcolor{LightGray}
		TS-RP1		
		& RPF1 
		& Verifica che durante la ricerca, l'utente abbia la possibilità di impostare dei filtri per raffinarla. 		
		& non implementato\\ \hline
		\rowcolor{white}
		TS-RD4		
		& RDF4 
		& Verifica che dopo una ricerca, l'utente venga avvisato con un messaggio nel caso in cui il sistema non abbia trovato nessun risultato corrispondente ai criteri selezionati.
		& non implementato\\ \hline
		\rowcolor{LightGray}
		TS-RP2		
		& RPF2 
		& Verifica che l'amministratore possa eliminare un utente iscritto alla piattaforma.
		& non implementato\\ \hline
		\rowcolor{white}
		TS-RP3		
		& RPF3 
		& Verifica che l'insegnante possa modificare una soluzione di un esercizio da lui fornita
		& non implementato\\ \hline
		\rowcolor{LightGray}
		TS-RD5		
		& RDF5 
		& Verifica che l'insegnante, accedendo alla sua area del profilo, possa visualizzare la lista degli esercizi da lui creati. 
		& non implementato\\ \hline
		\rowcolor{white}
		TS-RP4		
		& RPF4 
		& Verifica che l'insegnante possa eliminare una soluzione di un esercizio da lui fornita. 
		& non implementato\\ \hline
		\rowcolor{LightGray}
		TS-RP5		
		& RPF5 
		& Verifica che l'insegnante possa indicare gli argomenti trattati nell'esercizio in fase di creazione.
		& non implementato\\ \hline
		\rowcolor{white}
		TS-RO4		
		& ROF4 
		& Verifica che l'allievo possa inserire una frase da svolgere o selezionare un esercizio da quelli disponibili sul sistema.
		& non implementato\\ \hline
		\rowcolor{LightGray}
		TS-RO5		
		& ROF5 
		& Verifica che l'allievo possa compilare i campi relativi alle parole della frase al fine di completare l'esercizio selezionato.
		& non implementato\\ \hline
		\rowcolor{white}
		TS-RD6		
		& RDF6 
		& Verifica che lo sviluppatore possa ricercare le annotazioni di una particolare frase.
		& non implementato\\ \hline
		\rowcolor{LightGray}
		TS-RP6		
		& RPF6 
		& Verifica che lo sviluppatore possa ordinare la lista dei risultati ottenuti dalla ricerca tramite determinati parametri. 
		& non implementato\\ \hline
		\rowcolor{white}
		TS-RP7		
		& RPF7 
		& Verifica che l'amministratore possa eliminare uno qualsiasi degli esercizi inseriti nel sistema.
		& non implementato\\ \hline
		\rowcolor{LightGray}
		TS-RO6		
		& ROF6 
		& Verifica che l'insegnante possa inserire un esercizio nel sistema, indicando una soluzione per esso. 
		& non implementato\\ \hline
		\rowcolor{white}
		TS-RO7		
		& ROF7 
		& Verifica che l'insegnante possa inserire la soluzione dell'esercizio che sta creando; può renderla pubblica o privata. 
		& non implementato\\ \hline
		\rowcolor{LightGray}
		TS-RO8		
		& ROF8 
		& Verifica che l'allievo possa svolgere un esercizio da lui indicato e visualizzarne la relativa valutazione. 
		& non implementato\\ \hline
		\rowcolor{white}
		TS-RD7		
		& RDF7 
		& Verifica che l'allievo, accedendo al proprio profilo, possa visualizzare dei dati relativi ai propri progressi. 
		& non implementato\\ \hline
		\rowcolor{LightGray}
		TS-RD8		
		& RDF8 
		& Verifica che lo sviluppatore possa filtrare i dati trovati durante la ricerca ottenendo una lista di esercizi. 
		& non implementato\\ \hline
		\rowcolor{white}
		TS-RD9
		& RDF9 
		& Verifica che lo sviluppatore possa impostare un filtro temporale per la ricerca degli esercizi.
		& non implementato\\ \hline
		\rowcolor{LightGray}
		TS-RD10		
		& RDF10 
		& Verifica che lo sviluppatore possa includere o escludere dalla ricerca uno o più utenti. 
		& non implementato\\ \hline 
		\rowcolor{white}
		TS-RP8		
		& RPF8 
		& Verifica che lo sviluppatore possa visualizzare i dati relativi ad una particolare annotazione. 		
		& non implementato\\ \hline
		\rowcolor{LightGray}
		TS-RP9		
		& RPF9 
		& Verifica che lo sviluppatore possa visualizzare lo storico delle annotazioni. 
		&  non implementato\\ \hline
		\rowcolor{white}
		TS-RO9
		& ROF9 
		& Verifica che lo sviluppatore deve poter scaricare un file contenente i dati relativi agli esercizi ottenuti con la ricerca.
		& non implementato\\ \hline
		\rowcolor{LightGray}
		TS-RP10		
		& RPF10 
		& Verifica che lo sviluppatore può visualizzare le informazioni relative ad uno dei modelli disponibili. 
		& non implementato\\ \hline
		\rowcolor{white}
		TS-RP11		
		& RPF11 
		& Verifica che lo sviluppatore deve poter scaricare le informazioni riguardanti un modello. 
		& non implementato\\ \hline
		\rowcolor{LightGray}
		TS-RP12		
		& RPF12 
		& Verifica che lo sviluppatore deve poter creare un modello tramite la piattaforma. 
		& non implementato\\ \hline	
		
		\rowcolor{white}
		TS-RO10	
		& ROF10 
		& Verifica che l'insegnante possa creare una nuova classe. 
		& non implementato\\ \hline
		\rowcolor{LightGray}
		TS-RO11	
		& ROF11 
		& Verifica che l'insegnante possa eliminare una classe dal sistema. 
		& non implementato\\ \hline
		\rowcolor{white}
		TS-RO12
		& ROF12 
		& Verifica che l'insegnante possa aggiungere degli alunni ad una classe. 
		& non implementato\\ \hline
		\rowcolor{LightGray}
		TS-RO13
		& ROF13 
		& Verifica che l'insegnante possa aggiungere degli esercizi a quelli assegnati ad una classe. 
		& non implementato\\ \hline
		\rowcolor{white}
		TS-RP13
		& RPF13 
		& Verifica che l'insegnante possa visualizzare i progressi degli alunni di una propria classe.
		& non implementato\\ \hline
		\rowcolor{LightGray}
		TS-RO14
		& ROF14 
		& Verifica che l'insegnante possa eliminare un alunno dalla lista di quelli iscritti ad una delle proprie classi. 
		& non implementato\\ \hline
		\rowcolor{white}
		TS-RO15	
		& ROF15 
		& Verifica che l'insegnante possa visualizzare la lista degli alunni iscritti ad una delle sue classi. 
		& non implementato\\ \hline
		\rowcolor{LightGray}
		TS-RO16
		& ROF16 
		& Verifica che l'utente possa visualizzare la lista delle proprie classi. 
		& non implementato\\ \hline
		\rowcolor{white}
		TS-RD11	
		& RDF11 
		& Verifica che l'utente possa confermare le operazioni.
		& non implementato\\ \hline
		
		\caption{Test di sistema}
\end{longtable}}
\subsection{Processi}
\subsubsection{Studio di fattibilità}


\renewcommand{\arraystretch}{1.5}%
\begin{longtable}{|p{3.125cm}|p{3.125cm}|p{3.125cm}|p{3.125cm}|>{\centering\arraybackslash}m{2cm}|}
	\rowcolor{LightBlue}
	\multicolumn{4}{p{13.825cm}}{\centering\textbf{\textcolor{white}{Attributi}}}
		& 
			\textbf{\textcolor{white}{Grado}}
	 \\
		
	\rowcolor{LightBlue}
		\textbf{\textcolor{white}{N \newline not\newline implemented}}
		& \textbf{\textcolor{white}{P\newline partial\newline implemented}}
		& \textbf{\textcolor{white}{L\newline largely\newline implemented}} 
		& \textbf{\textcolor{white}{F\newline fully\newline implemented}} 
		& \\ \hline

		
		\rowcolor{LightGray}
		Process\newline optimization & Process deployment & &Process Performance & Livello 2 Managed\\
		\rowcolor{white}
		& Process\newline measurement & & Process management & \\
		\rowcolor{LightGray}
		& Process control & & Work product\newline management & \\
		\rowcolor{white}
		& Process innovation & & Process definition & \\ \hline
		
		\caption{Processi di avvio}
\end{longtable}


\subsubsection{Norme di progetto}
{\renewcommand{\arraystretch}{1.5}%
\begin{longtable}{|p{3.125cm}|p{3.125cm}|p{3.125cm}|p{3.125cm}|>{\centering\arraybackslash}m{2cm}|}
	\rowcolor{LightBlue}
	\multicolumn{4}{p{13.825cm}}{\centering\textbf{\textcolor{white}{Attributi}}}
		& \textbf{\textcolor{white}{Grado}}\\
		
	\rowcolor{LightBlue}
		\textbf{\textcolor{white}{N \newline not\newline implemented}}
		& \textbf{\textcolor{white}{P\newline partial\newline implemented}}
		& \textbf{\textcolor{white}{L\newline largely\newline implemented}} 
		& \textbf{\textcolor{white}{F\newline fully\newline implemented}} 
		& \\ \hline
		
		\rowcolor{LightGray}
		Process optimization & Process innovation & Process measurement & Process performance & Livello 3\newline Established \\
		\rowcolor{white}
		& & Process control & Process management & \\
		\rowcolor{LightGray}
		& & &  Work product\newline management & \\
		\rowcolor{white}
		& & & Process definition & \\
		\rowcolor{LightGray}
		& & & Process deployment & \\ \hline
		
		\caption{Processi di analisi di sistema}
\end{longtable}
}

\subsubsection{Pianificazione progetto}
{\renewcommand{\arraystretch}{1.5}%
	\begin{longtable}{|p{3.125cm}|p{3.125cm}|p{3.125cm}|p{3.125cm}|>{\centering\arraybackslash}m{2cm}|}
	\rowcolor{LightBlue}
	\multicolumn{4}{p{13.825cm}}{\centering\textbf{\textcolor{white}{Attributi}}}
		& \textbf{\textcolor{white}{Grado}}\\
		
	\rowcolor{LightBlue}
		\textbf{\textcolor{white}{N \newline not\newline implemented}}
		& \textbf{\textcolor{white}{P\newline partial\newline implemented}}
		& \textbf{\textcolor{white}{L\newline largely\newline implemented}} 
		& \textbf{\textcolor{white}{F\newline fully\newline implemented}} 
		& \\
		\hline
		\rowcolor{LightGray}
		Process\newline optimization & Process innovation & Process control & Process performance & Livello 3\newline Established \\
		\rowcolor{white}
		&&& Performance\newline management& \\
		\rowcolor{LightGray}
		&&& Work product\newline management& \\
		\rowcolor{white}
		&&& Process definition & \\
		\rowcolor{LightGray}
		&&& Process deployment & \\
		\rowcolor{white}
		&&& Process\newline measurement	& \\ \hline

		\caption{Processi di analisi software e progettazione}
\end{longtable}
}
\subsubsection{Pianificazione qualifica}
{\renewcommand{\arraystretch}{1.5}%
	\begin{longtable}{|p{3.125cm}|p{3.125cm}|p{3.125cm}|p{3.125cm}|>{\centering\arraybackslash}m{2cm}|}
	\rowcolor{LightBlue}
	\multicolumn{4}{p{13.825cm}}{\centering\textbf{\textcolor{white}{Attributi}}}
		& \textbf{\textcolor{white}{Grado}}\\
		
	\rowcolor{LightBlue}
		\textbf{\textcolor{white}{N \newline not\newline implemented}}
		& \textbf{\textcolor{white}{P\newline partial\newline implemented}}
		& \textbf{\textcolor{white}{L\newline largely\newline implemented}} 
		& \textbf{\textcolor{white}{F\newline fully\newline implemented}} 
		& \\ \hline
		\rowcolor{LightGray}
		Process\newline optimization & Process innovation & Process control & Processo performance & Livello 3 \newline Established\\
		\rowcolor{white}
		 &  &  & Performance\newline management & \\
		\rowcolor{LightGray}
		 &  &  & Work Product\newline management & \\
		\rowcolor{white}
		 &  &  & Process definition & \\
		\rowcolor{LightGray}
		 &  &  & Process deployment & \\
		\rowcolor{white}
		 &  &  & Process\newline measurement & \\ \hline
		\caption{Processi di realizzazione}
\end{longtable}
}

\subsubsection{Analisi dei requisiti}
{\renewcommand{\arraystretch}{1.5}%
	\begin{longtable}{|p{3.125cm}|p{3.125cm}|p{3.125cm}|p{3.125cm}|>{\centering\arraybackslash}m{2cm}|}
	\rowcolor{LightBlue}
	\multicolumn{4}{p{13.825cm}}{\centering\textbf{\textcolor{white}{Attributi}}}
		& \textbf{\textcolor{white}{Grado}}\\
		
	\rowcolor{LightBlue}
		\textbf{\textcolor{white}{N \newline not\newline implemented}}
		& \textbf{\textcolor{white}{P\newline partial\newline implemented}}
		& \textbf{\textcolor{white}{L\newline largely\newline implemented}} 
		& \textbf{\textcolor{white}{F\newline fully\newline implemented}} 
		& \\ \hline
		
		\rowcolor{LightGray}
		Process\newline optimization & Process innovation & Process control & Processo performance & Livello 3 \newline Established\\
		\rowcolor{white}
		&  &  & Performance\newline management & \\
		\rowcolor{LightGray}
		&  &  & Work Product\newline management & \\
		\rowcolor{white}
		&  &  & Process definition & \\
		\rowcolor{LightGray}
		&  &  & Process deployment & \\
		\rowcolor{white}
		&  &  & Process\newline measurement & \\ \hline
		\caption{Processi di validazione}
\end{longtable}
}

