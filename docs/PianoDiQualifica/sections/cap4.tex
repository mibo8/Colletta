\section{Resoconto attività di verifica}
\subsection{Prodotto}
\subsubsection{Documentazione}
Nella tabella seguente vengono riportati i risultati delle verifiche eseguite sui documenti. Il resoconto contiene le verifiche sia dei documenti esterni, cioè utili al committente, sia interni, utili invece al team Ottobit.\\
\begin{longtable}{p{3cm} p{4cm} p{5cm} p{2cm}}
	\rowcolor{LightBlue}
		  \textbf{\textcolor{white}{Data}}
		& \textbf{\textcolor{white}{Autore}}
		& \textbf{\textcolor{white}{Documento}} 
		& \textbf{\textcolor{white}{Versione}}\\
		2018-12-28
		& Giovanni Bergo
		& \textit{Piano di qualifica}
		& 0.1.0\\
		\rowcolor{LightGray}
		\multicolumn{4}{p{15.25cm}}{\textbf{Descrizione:} 
		Footer mancante. Piccoli errori grammaticali e di punteggiatura. Completamento descrizione §1.3 Note esplicative. Url citazioni non collegati al sito web. Introdurre rifermento a sezioni come indicato nelle 	
		textit{Norme di progetto}. Elenco delle tabelle e figure mancanti.
		}\\
		\rowcolor{LightGray}
		\multicolumn{4}{p{15.25cm}}{
			\textbf{Indice di Gullpease:} 66
		}\\
		\rowcolor{LightGray}
		\multicolumn{4}{p{15.25cm}}{
			\textbf{Esito:} Non accettato
		}\\
		\hline
		
		2018-12-26
		& Michele Bortone
		& \textit{Norme di Progetto}
		& 0.2.0\\
	\rowcolor{LightGray}
	\multicolumn{4}{p{15.25cm}}{\textbf{Descrizione:} 
	Documento conforme e senza particolari errori da evidenziare.
	Pronto per l'approvazione.
	}\\
	\rowcolor{LightGray}
	\multicolumn{4}{p{15.25cm}}{
	\textbf{Indice di Gullpease:} 81
	}\\
	\rowcolor{LightGray}
	\multicolumn{4}{p{15.25cm}}{
	\textbf{Esito:} Accettato
	}\\
	\hline
		2018-12-18
		& Giovanni Peron
		& \textit{Piano di progetto}
		& 0.2.0\\
		\rowcolor{LightGray}
	\multicolumn{4}{p{15.25cm}}{\textbf{Descrizione:} Nulla da segnalare.
	}\\
	\rowcolor{LightGray}
	\multicolumn{4}{p{15.25cm}}{
	\textbf{Indice di Gullpease:} 86
	}\\
		\rowcolor{LightGray}
	\multicolumn{4}{p{15.25cm}}{
	\textbf{Esito:} Accettato
	}\\
		\hline
				2018-12-16
		& Michele Bortone
		& \textit{Studio di fattibilità}
		& 0.2.0\\
		\rowcolor{LightGray}
	\multicolumn{4}{p{15.25cm}}{\textbf{Descrizione:} 
	Nulla da segnalare.
	}\\
	\rowcolor{LightGray}
	\multicolumn{4}{p{15.25cm}}{
	\textbf{Indice di Gullpease:} 60
	}\\
		\rowcolor{LightGray}
	\multicolumn{4}{p{15.25cm}}{
	\textbf{Esito:} Accettato
	}\\
	\hline
		2018-12-16
		& Giovanni Peron
		& \textit{Piano di progetto}
		& 0.1.0\\
		\rowcolor{LightGray}
	\multicolumn{4}{p{15.25cm}}{\textbf{Descrizione:} Nella tabella del'analisi dei rischi del capitolo §2 ci sono ripetizioni nelle righe R01 e T01 entrambe nella colonna rilevamento. Il contenuto della tabella risulta tagliato a fine pagina 4. In §3.1 a riga 6 suggerisco di inserire Proof of Concept nel glossario. In tutto il documento rivedere il formato delle date secondo le norme di progetto. Per informazioni più dettagliate vedi i commenti scritti nel file relativo al documento.
	}\\
	\rowcolor{LightGray}
	\multicolumn{4}{p{15.25cm}}{
	\textbf{Indice di Gullpease:} 95
	}\\
		\rowcolor{LightGray}
	\multicolumn{4}{p{15.25cm}}{
	\textbf{Esito:} Non accettato
	}\\
		\hline
		2018-12-16
		& Michele Bortone
		& \textit{Norme di progetto}
		& 0.1.0\\
		\rowcolor{LightGray}
	\multicolumn{4}{p{15.25cm}}{\textbf{Descrizione:} 
	Il correttore segnala alcuni errori ortografici.
	Sezione §1 incompleta.
	Sezione §2.1 incompleta.
	Nella sezione §4.1.3.5 è presente un errore nella descrizione delle norme riguardanti l'inserimento delle figure all'interno di un documento. Suggerisco di aggiungere l'obbligo di inserire una breve didascalia dell'immagine corrispondente.
	Nella sezione §5.1.4 è presente un errore riguardo i compiti di ciascun ruolo. Bisogna correggere il redattore dello Studio di fattibilità.
	}\\
	\rowcolor{LightGray}
	\multicolumn{4}{p{15.25cm}}{
	\textbf{Indice di Gullpease:}79
	}\\
		\rowcolor{LightGray}
	\multicolumn{4}{p{15.25cm}}{
	\textbf{Esito:} Non accettato
	}\\
	\hline
		2018-12-16
		& Michele Bortone
		& \textit{Studio di fattibilità}
		& 0.1.0\\
	\rowcolor{LightGray}
	\multicolumn{4}{p{15.25cm}}{\textbf{Descrizione:} 
	Il correttore segnala alcuni errori ortografici.
	Elenchi puntati non conformi alle \textit{Norme di progetto}.
	}\\
	\rowcolor{LightGray}
	\multicolumn{4}{p{15.25cm}}{
	\textbf{Indice di Gullpease:} 58
	}\\
		\rowcolor{LightGray}
	\multicolumn{4}{p{15.25cm}}{
	\textbf{Esito:} Non accettato
	}\\\hline
	\caption{Resoconto attività di verifica}
\end{longtable}

\subsection{Processi}
\subsubsection{Studio di fattibilità}
%non riesco a fare la tabella proprio come vorrei, macherebbe un multirow credo sul grado
\begin{longtable}{|p{3.125cm}|p{3.125cm}|p{3.125cm}|p{3.125cm}|p{1.5cm}|}
	\rowcolor{LightBlue}
	\multicolumn{4}{p{13.825cm}}{\centering\textbf{\textcolor{white}{Attributi}}}
		& \textbf{\textcolor{white}{Grado}}\\
		
	\rowcolor{LightBlue}
		\textbf{\textcolor{white}{N \newline not\newline implemented}}
		& \textbf{\textcolor{white}{P\newline partial\newline implemented}}
		& \textbf{\textcolor{white}{L\newline largely\newline implemented}} 
		& \textbf{\textcolor{white}{F\newline fully\newline implemented}} 
		& \\

		\hline
		\rowcolor{LightGray}
		 Process optimization
		& Process deployment
		\newline Process measurement
		\newline Process control
		\newline Process innovation
		& 
		&  Process performance
		\newline Perfomance management
		\newline Work product management
		\newline Process definition
		& Livello 2\\[0.5cm]
		\caption{Processi di avvio}
\end{longtable}
[*] Guardando la lista dei 6 livelli nell'appendice A\\(0 - Incompleted, 1 - Performed, 2 - Managed, 3 - Estabilished, 4 - Predictable, 5 - Optimizing)\\ in base agli attributi F totalmente implementati sarà possibile scegliere uno tra questi 6 livelli di qualità del processo.
\subsubsection{Norme di progetto}
\begin{longtable}{|p{3.125cm}|p{3.125cm}|p{3.125cm}|p{3.125cm}|p{1.5cm}|}
	\rowcolor{LightBlue}
	\multicolumn{4}{p{13.825cm}}{\centering\textbf{\textcolor{white}{Attributi}}}
		& \textbf{\textcolor{white}{Grado}}\\
		
	\rowcolor{LightBlue}
		\textbf{\textcolor{white}{N \newline not\newline implemented}}
		& \textbf{\textcolor{white}{P\newline partial\newline implemented}}
		& \textbf{\textcolor{white}{L\newline largely\newline implemented}} 
		& \textbf{\textcolor{white}{F\newline fully\newline implemented}} 
		& \\

		\hline
		\rowcolor{LightGray}
		Process optimization
		& Process innovation
		& Process measurement
		\newline Process control
		& Process performance
		\newline Perfomance management
		\newline Work product management
		\newline Process definition
		\newline Process deployment
		& Livello 3\\[0.5cm]
		\caption{Processi di analisi di sistema}
\end{longtable}
[*] Guardando la lista dei 6 livelli nell'appendice A\\(0 - Incompleted, 1 - Performed, 2 - Managed, 3 - Estabilished, 4 - Predictable, 5 - Optimizing)\\ in base agli attributi F totalmente implementati sarà possibile scegliere uno tra questi 6 livelli di qualità del processo.
\subsubsection{Pianificazione progetto}
\begin{longtable}{|p{3.125cm}|p{3.125cm}|p{3.125cm}|p{3.125cm}|p{1.5cm}|}
	\rowcolor{LightBlue}
	\multicolumn{4}{p{13.825cm}}{\centering\textbf{\textcolor{white}{Attributi}}}
		& \textbf{\textcolor{white}{Grado}}\\
		
	\rowcolor{LightBlue}
		\textbf{\textcolor{white}{N \newline not\newline implemented}}
		& \textbf{\textcolor{white}{P\newline partial\newline implemented}}
		& \textbf{\textcolor{white}{L\newline largely\newline implemented}} 
		& \textbf{\textcolor{white}{F\newline fully\newline implemented}} 
		& \\

		\hline
		\rowcolor{LightGray}
		Process optimization
		& Process innovation
		& Process control
		& Process performance\newline  Performance management\newline  Work product management\newline  
Process definition\newline  Process deployment\newline  Process measurement
		&  Livello 3 %\newline Estabilished
		\\[0.5cm]
		\caption{Processi di analisi software e progettazione}
\end{longtable}

\subsubsection{Pianificazione qualifica}
\begin{longtable}{|p{3.125cm}|p{3.125cm}|p{3.125cm}|p{3.125cm}|p{1.5cm}|}
	\rowcolor{LightBlue}
	\multicolumn{4}{p{13.825cm}}{\centering\textbf{\textcolor{white}{Attributi}}}
		& \textbf{\textcolor{white}{Grado}}\\
		
	\rowcolor{LightBlue}
		\textbf{\textcolor{white}{N \newline not\newline implemented}}
		& \textbf{\textcolor{white}{P\newline partial\newline implemented}}
		& \textbf{\textcolor{white}{L\newline largely\newline implemented}} 
		& \textbf{\textcolor{white}{F\newline fully\newline implemented}} 
		& \\

		\hline
		\rowcolor{LightGray}
		Lista attributi non implementati
		& Lista attributi \newline parzialmente implementati
		& Lista attributi in gran parte \newline implementati
		& Lista attributi \newline totalmente \newline implementati
		& Livello 0\newline vedi[*]\\[0.5cm]
		\caption{Processi di realizzazione}
\end{longtable}
[*] Guardando la lista dei 6 livelli nell'appendice A\\(0 - Incompleted, 1 - Performed, 2 - Managed, 3 - Estabilished, 4 - Predictable, 5 - Optimizing)\\ in base agli attributi F totalmente implementati sarà possibile scegliere uno tra questi 6 livelli di qualità del processo.
\subsubsection{Analisi dei requisiti}
\begin{longtable}{|p{3.125cm}|p{3.125cm}|p{3.125cm}|p{3.125cm}|p{1.5cm}|}
	\rowcolor{LightBlue}
	\multicolumn{4}{p{13.825cm}}{\centering\textbf{\textcolor{white}{Attributi}}}
		& \textbf{\textcolor{white}{Grado}}\\
		
	\rowcolor{LightBlue}
		\textbf{\textcolor{white}{N \newline not\newline implemented}}
		& \textbf{\textcolor{white}{P\newline partial\newline implemented}}
		& \textbf{\textcolor{white}{L\newline largely\newline implemented}} 
		& \textbf{\textcolor{white}{F\newline fully\newline implemented}} 
		& \\

		\hline
		\rowcolor{LightGray}
		Lista attributi non implementati
		& Lista attributi \newline parzialmente implementati
		& Lista attributi in gran parte \newline implementati
		& Lista attributi \newline totalmente \newline implementati
		& Livello 0\newline vedi[*]\\[0.5cm]
		\caption{Processi di validazione}
\end{longtable}
[*] Guardando la lista dei 6 livelli nell'appendice A\\(0 - Incompleted, 1 - Performed, 2 - Managed, 3 - Estabilished, 4 - Predictable, 5 - Optimizing)\\ in base agli attributi F totalmente implementati sarà possibile scegliere uno tra questi 6 livelli di qualità del processo.

