\appendix
\section{Specifica dei test}
\subsection{Test di validazione}
Questo tipo di test verrà utilizzato durante l'attività di collaudo del prodotto finale così da accertare che esso soddisfi le richieste del committente. \\
Per ogni test viene specificato il proprio codice univoco, la descrizione, lo stato di implementazione attuale ed il codice identificativo del requisito -o dei requisiti- ad esso associato. 

	\begin{longtable}{|>{\centering\arraybackslash}m{1.6cm}|>{\centering\arraybackslash}m{6.41cm}|>{\centering\arraybackslash}m{3.1cm} | >{\centering\arraybackslash}m{2.6cm}|}		
		\rowcolor{LightBlue}
		\textbf{\textcolor{white}{Codice}}
		& \multicolumn{1}{|c|}{\textbf{\textcolor{white}{ Descrizione}}}
		& \textbf{\textcolor{white}{Requisito associato}}\\
		TV-RO1 & L'utente non riconosciuto vuole registrarsi alla piattaforma come allievo, creando un account personale. All'utente è richiesto di:
		\begin{itemize}
			\item Raggiungere la pagina di registrazione;
			\item Inserire i dati richiesti quali:
				\begin{itemize}
				 	\item nome;
				 	\item cognome;
				 	\item username;
				 	\item email;
				 	\item password;
				 	\item nome della scuola di appartenenza;
				 	\item città sede della scuola di appartenenza;
				\end{itemize}
			\item Confermare la registrazione.
		\end{itemize} & ROF1, ROF24, ROF25, ROF26, ROF27, ROF28, ROF29, ROF30 \\ \hline
		\rowcolor{LightGray}
		TV-RO2 & L'utente non riconosciuto vuole registrarsi alla piattaforma come insegnante, creando un account personale. All'utente è richiesto di:
		\begin{itemize}
			\item Raggiungere la pagina di registrazione;
			\item Inserire i dati richiesti quali:
				\begin{itemize}
				 	\item nome;
				 	\item cognome;
				 	\item username;
				 	\item email;
				 	\item password;
				 	\item nome della scuola di appartenenza;
				 	\item città sede della scuola di appartenenza;
				 	\item codice INPS;
				\end{itemize}
			\item Confermare la registrazione.
		\end{itemize} & ROF22, ROF31, ROF24, ROF25, ROF26, ROF27, ROF28, ROF29, ROF30 \\ \hline
		TV-RO3 & L'utente non riconosciuto vuole eseguire l'accesso alla piattaforma tramite le sue credenziali. All'utente è richiesto di: 
		\begin{itemize}
			\item Raggiungere la pagina di autenticazione;
			\item Inserire il proprio username e password;
			\item Confermare l'accesso.
		\end{itemize}
		& ROF2, ROF33, ROF34 \\ \hline
		  \rowcolor{LightGray}
		TV-RD1 & L'utente vuole modificare i dati del proprio profilo personale. All'utente è richiesto di:
		\begin{itemize}
			\item Raggiungere la pagina di modifica dei dati dal profilo;
			\item Modificare una o più voci tra username, password, scuola e città;
			\item Confermare la modifica.
		\end{itemize} & RDF1, RDF23, RDF24, RDF25, RDF26\\ \hline
		TV-RD2 & Il  moderatore vuole verificare le credenziali di un utente che vuole registrarsi come insegnante. Al moderatore è richiesto di:
		\begin{itemize}
			\item Raggiungere la pagina di amministrazione della piattaforma;
			\item Visualizzare la lista degli utenti che richiedono i  ruolo di insegnante;
			\item Il moderatore seleziona uno degli utenti;
			\item Il moderatore accetta o rifiuta l'utente selezionato.
		\end{itemize} & RDF2 \\ \hline
		  \rowcolor{LightGray}
		TV-RD3 & L'utente vuole registrarsi alla piattaforma. 
		\begin{itemize}
			\item L'utente inserisce delle credenziali errate;
			\item L'utente visualizza un messaggio di errore.
		\end{itemize} & RDF3 \\ \hline
		TV-RO4 & L'utente vuole cercare degli esercizi sulla piattaforma. All'utente è richiesto di:
		\begin{itemize}
			\item Raggiungere la pagina di ricerca degli esercizi;
			\item Scrivere una frase o parte di essa nella barra di ricerca;
			\item Selezionare, eventualmente, dei filtri per la ricerca;
			\item Avviare la ricerca.
		\end{itemize}  & ROF3 RPF1 RPF2 RPF3 RPF4\\ \hline
		  \rowcolor{LightGray}
		TV-RD4 & L'allievo vuole visualizzare la lista delle classi a cui appartiene. All'allievo è richiesto di: 
		\begin{itemize}
			\item Accedere alla pagina dedicata al proprio profilo;
			\item Selezionare l'opzione "lista delle classi".
		\end{itemize} & RDF4 \\ \hline
		TV-RP1 & Il moderatore vuole eliminare un utente iscritto alla piattaforma. Al moderatore è richiesto di: 
		\begin{itemize}
			\item Visualizzare la lista degli utenti iscritti nella piattaforma;
			\item Indicare uno o più utenti da eliminare;
			\item Confermare l'eliminazione degli utenti  selezionati.
		\end{itemize}  & RPF5 \\ \hline
		  \rowcolor{LightGray}
		TV-RP2 & L'insegnante vuole modificare una soluzione di un esercizio da lui fornita. All'insegnante è richiesto di: 
		\begin{itemize}
			\item Selezionare la lista degli esercizi da lui inseriti;
			\item Visualizzare la precedente soluzione;
			\item Modificare le classi grammaticali della precedente soluzione;
			\item Confermare la modifica.
		\end{itemize}  & RPF6 \\ \hline
		TV-RD5 & L'insegnante vuole visualizzare la lista degli esercizi da lui creati. All'insegnante è richiesto di: 
		\begin{itemize}
			\item Accedere all'area del suo profilo;
			\item Selezionare la voce "esercizi inseriti".
		\end{itemize}  & RDF5 \\ \hline
		  \rowcolor{LightGray}
		TV-RP3 & L'insegnante vuole eliminare una soluzione di un esercizio da lui fornita. All'insegnante è richiesto di:
		\begin{itemize}
			\item Visualizzare la lista degli esercizi da lui inseriti;
			\item Selezionare una o più soluzioni da eliminare;
			\item Confermare l'eliminazione
		\end{itemize}  & RPF7 \\ \hline
		TV-RO5 & L'insegnante vuole inserire un esercizio nel sistema. All'insegnante è richiesto di: 
		\begin{itemize}
			\item Raggiungere la pagina di inserimento di un nuovo esercizio;
			\item Inserire una frase;
			\item Inserire una soluzione;
			\item Specificare gli argomenti;
			\item Specificare la difficoltà;
			\item Specificare se la soluzione è pubblica o privata;
			\item Confermare l'inserimento.
		\end{itemize}  & ROF4 ROF6 RPF8 RPF9\\ \hline
		  \rowcolor{LightGray}
		TV-RO6 & L'utente vuole svolgere un esercizio da lui indicato. All'utente è richiesto di:
		\begin{itemize}
			\item Raggiungere la pagina per l'esecuzione degli esercizi;
			\item Inserire una frase da svolgere o selezionare un esercizio dal sistema;
			\item Compilare i campi;
			\item Confermare i dati inseriti;
			\item Visualizzare la soluzione.
		\end{itemize}  & ROF6 ROF7 ROF8\\ \hline
		TV-RD6 & L'allievo vuole visualizzare i dati relativi ai propri progressi. All'allievo è richiesto di:
		\begin{itemize}
			\item Accedere alla vista del proprio profilo;
			\item Visualizzare i dati relativi ai propri progressi.
		\end{itemize}  & RDF6 \\ \hline
		  \rowcolor{LightGray}
		TV-RD7 & Lo sviluppatore vuole visualizzare la lista delle annotazioni di una particolare frase. Allo sviluppatore è richiesto di: 
		\begin{itemize}
			\item Accedere all'area dati;
			\item Scrivere una frase o parte di essa all'interno della barra di ricerca;
			\item Selezionare eventualmente i filtri di ricerca;
			\item Confermare la ricerca.
		\end{itemize}  & RDF7 RDF8 RDF9 RDF10 RDF15 \\ \hline
		TV-RP4 & Lo sviluppatore vuole visualizzare i dati relativi ad una particolare annotazione. Allo sviluppatore è richiesto di:
		\begin{itemize}
			\item Visualizzare la lista delle annotazioni ricercate;
			\item Selezionare un'annotazione;
			\item Visualizzare data, utente e soluzione proposta.
		\end{itemize}  & RPF10 \\ \hline
		  \rowcolor{LightGray}
		TV-RP5 & Lo sviluppatore vuole visualizzare lo storico delle annotazioni. Allo sviluppatore è richiesto di:
		\begin{itemize}
			\item Visualizzare la lista delle annotazioni cercate;
		\end{itemize}  & RPF11 \\ \hline
		TV-RP6 & Lo sviluppatore vuole ordinare la lista dei risultati ottenuti dalla ricerca. Allo sviluppatore è richiesto di: 
		\begin{itemize}
			\item Visualizzare la lista delle annotazioni ricercate;
			\item Scegliere un parametro secondo il quale ordinare i risultati;
			\item Confermare l'ordinamento scelto.
		\end{itemize}  & RPF12 \\ \hline
		  \rowcolor{LightGray}
		TV-RO6 & Lo sviluppatore vuole scaricare un file contenente i dati relativi agli esercizi ottenuti con la ricerca. Allo sviluppatore è richiesto di:
		\begin{itemize}
			\item Visualizzare la lista delle annotazioni ricercate;
			\item Richiedere il download dei dati;
			\item Scegliere il path per il salvataggio del file;
			\item Confermare ed eseguire il salvataggio.
		\end{itemize}  & ROF9 \\ \hline
		TV-RP7 & Lo sviluppatore vuole visualizzare le informazioni relative ad un dataset. Allo sviluppatore è richiesto di: 
		\begin{itemize}
			\item Visualizzare la lista delle annotazioni ricercate,
			contenente l'input della frase della fase di train del software di apprendimento automatico in base alla ricerca effettuata;
			\item Richiedere il download dei dati;
			\item Scegliere il path per il salvataggio del file;
			\item Confermare ed eseguire il salvataggio.
		\end{itemize}  & RPF13 \\ \hline
		  \rowcolor{LightGray}
		TV-RP8 & Lo sviluppatore vuole scaricare un modello. Allo sviluppatore è richiesto di:
		\begin{itemize}
			\item Visualizzare la lista dei modelli disponibili per l'applicazione;
			\item Selezionare un modello;
			\item Selezionare l'opzione "download";
			\item Scegliere il path per il salvataggio del file;
			\item Confermare ed eseguire il salvataggio.			
		\end{itemize}  & RPF14 \\ \hline
		
		TV-RP9 & Lo sviluppatore creare un modello alla piattaforma. Allo sviluppatore è richiesto di:
\begin{itemize}
 \item Selezionare la voce ”Aggiungi modello”;
 \item Inserire un file contenente il dataset;
 \item Confermare la creazione.
\end{itemize}  & RPF15 \\ \hline

		  \rowcolor{LightGray}
TV-RP10 & Il moderatore vuole eliminare un esercizio. Al moderatore è richiesto di:
\begin{itemize}
 \item Indicare gli esercizi da eliminare;
 \item Confermare l'eliminazione dell'esercizio.
\end{itemize}  & RPF16 \\ \hline

TV-RO7 & L'insegnante vuole creare una classe. All'insegnante è richiesto di:
\begin{itemize}
 \item Selezionare l'opzione "Crea classe";
 \item Assegnare un nome alla classe;
 \item Assegnare una descrizione alla classe;
 \item Confermare la creazione della classe.
\end{itemize}  & ROF10 \\ \hline

		  \rowcolor{LightGray}
TV-RO8 & L'insegnante vuole eliminare una classe. All'insegnante è richiesto di:
\begin{itemize}
 \item Cliccare sul pulsante di eliminazione della classe;
 \item Confermare l'eliminazione della classe.
\end{itemize}  & ROF11 \\ \hline

TV-RO9 & L'insegnante vuole aggiungere degli alunni ad una classe. All'insegnante è richiesto di:
\begin{itemize}
 \item Selezionare l'opzione "Aggiungi alunni";
 \item Visualizzare la lista di tutti gli alunni presenti sulla piattaforma;
 \item Scrivere l'username di un alunno, o una sua parte;
 \item Selezionare gli alunni da inserire;
 \item Confermare l'inserimento.
\end{itemize}  & ROF12, RPF17, RPF18 \\ \hline

		  \rowcolor{LightGray}
TV-RO10 & L'insegnante vuole assegnare degli esercizi ad una classe. All'insegnante è richiesto di:
\begin{itemize}
 \item Selezionare degli esercizi in lista da assegnare;
 \item Indicare la classe a cui assegnarli;
 \item Scrivere l'username di un alunno, o una sua parte;
 \item Confermare l'assegnazione.
\end{itemize}  & ROF13 \\ \hline

TV-RP11 & L'insegnante vuole visualizzare i progressi di un alunno della classe. All'insegnante è richiesto di:
\begin{itemize}
 \item Selezionare uno studente;
 \item Visualizzare i grafici che riportano la media totale, la media per tipologia di
 esercizi e lo sviluppo della media nel tempo.
\end{itemize}  & RPF19 \\ \hline
		
		  \rowcolor{LightGray}
	TV-RO11 & L'insegnante vuole eliminare un alunno dalla lista degli iscritti ad una classe. All'insegnante è richiesto di:
		\begin{itemize}
			\item Visualizzare la lista degli alunni della classe;
			\item Selezione l'allievo da rimuovere dalla classe;
			\item Confermare l'eliminazione.		
		\end{itemize}  & ROF14 \\ \hline	
		
		TV-RO12 & L'insegnante vuole visualizzare la lista degli alunni iscritti ad una delle sue classi. All'insegnante è richiesto di:
		\begin{itemize}
			\item Accedere alla pagina di gestione delle proprie classi;
			\item Selezionare la voce "vedi alunni".		
		\end{itemize}  & ROF15 \\ \hline
		
		  \rowcolor{LightGray}
		TV-RO13 & L'insegnante vuole visualizzare la lista delle proprie classi. All'insegnante è richiesto di:
		\begin{itemize}
			\item Accedere alla pagina del proprio profilo;
			\item Selezionare la voce "vedi classi".		
		\end{itemize}  & ROF16 \\ \hline
		
		TV-RD7 & L'allievo vuole annullare l'iscrizione ad una classe. Allo studente è richiesto di:
		\begin{itemize}
			\item Visualizzare la lista delle classi a cui appartiene;
			\item Selezionare l'opzione "disiscriviti".		
		\end{itemize}  & RDF11 \\ \hline
		
		  \rowcolor{LightGray}
		TV-RO14 & L'utente vuole autenticarsi alla piattaforma. 
		\begin{itemize}
			\item L'utente inserisce delle credenziali errate;
			\item L'utente visualizza un messaggio di errore.
		\end{itemize}  & RDF14 \\ \hline
		
		TV-RO15 & L'utente vuole modificare le proprie credenziali. 
		\begin{itemize}
			\item L'utente inserisce delle credenziali errate;
			\item L'utente visualizza un messaggio di errore.
		\end{itemize}  & RDF15 \\ \hline
		
		  \rowcolor{LightGray}
  TV-RP12 & L’allievo vuole selezionare un esercizio assegnato. All'allievo è richiesto di:
  \begin{itemize}
   \item Visualizzare le informazioni della classe indicata
   \item Selezionare un esercizio assegnato
  \end{itemize}  & RPF28 \\ \hline

 TV-RO16 & L’insegnante vuole accedere all’area di inserimento nuovo esercizio. All'insegnante è richiesto di:
 \begin{itemize}
  \item Raggiungere la vista principale dell’applicazione.
  \item Selezionare la voce ”Inserisci esercizio”
 \end{itemize}  & ROF23 \\ \hline
  \rowcolor{LightGray}
  TV-RD8 & L’utente riconosciuto vuole disconnettersi dalla piattaforma. All'utente è richiesto di:
  \begin{itemize}
   \item Selezionare l’opzione ”Logout” da qualsiasi vista dell'applicazione
  \end{itemize}  & RDF22 \\ \hline
 
 TV-RP13 & Lo sviluppatore vuole visualizzare informazioni riguardanti i modelli disponibili nella piattaforma. Allo sviluppatore è richiesto:
 \begin{itemize}
  \item  Visualizzare la lista dei modelli disponibili per l’applicazione;
  \item Selezionare un modello
  \item Selezionare l’opzione ”Visualizza informazioni”
 \end{itemize}  & RPF26 \\ \hline

\rowcolor{LightGray}
TV-RD9 & Lo sviluppatore vuole visualizzare la lista dei modelli disponibili.  Allo sviluppatore è richiesto di:

\begin{itemize}
 \item Raggiungere la vista dell’area dati.
 \item Selezionare la voce ”Lista dei modelli”
\end{itemize}  & RDF20 \\ \hline

TV-RD10 & L’utente riconosciuto vuole accedere alla vista del proprio profilo personale. All'utente è richiesto di:
\begin{itemize}
  \item Raggiungere la vista principale dell'applicazione;
 \item Selezionare la voce ”Profilo personale”.
\end{itemize}  & RDF19 \\ \hline

\rowcolor{LightGray}
TV-RD11 & L’insegnante vuole ricercare gli esercizi che ha inserito inserendo la frase o una parte di essa. All'insegnante è richiesto di:


\begin{itemize}
 \item Visualizzare la lista degli esercizi che ha inserito nella piattaforma;
 \item Scrivere nella barra di ricerca una frase o una sua parte.
\end{itemize}  & RDF18 \\ \hline

TV-RP14 & Il moderatore vuole ricercare gli utenti inserendo l’username o una parte di esso. Al moderatore è richiesto di:
\begin{itemize}
 \item Visualizzare la lista degli utenti presenti nella piattaforma;
 \item Scrivere nella barra di ricerca una stringa.
\end{itemize}  & RPF25 \\ \hline

		  \rowcolor{LightGray}
TV-RD12 & Il moderatore vuole visualizzare la lista degli utenti iscritti alla piattaforma. Al moderatore è richiesto di:
\begin{itemize}
 \item Raggiungere la vista di amministrazione dell’applicazione;
 \item Selezionare l’opzione ”Utenti”.
\end{itemize}  & RDF17 \\ \hline

TV-RP15 & Il moderatore vuole ricercare gli esercizi inserendo la frase o una parte di essa. Al moderatore è richiesto di:

\begin{itemize}
 \item Raggiungere la lista degli esercizi inseriti nella piattaforma.
 \item Scrivere nella barra di ricerca una frase o una sua parte.
\end{itemize}  & RPF24 \\ \hline

		  \rowcolor{LightGray}
TV-RD13 & Il moderatore vuole visualizzare la lista degli esercizi inseriti nella piattaforma. Al moderatore è richiesto di:

\begin{itemize}
 \item Raggiungere la vista di amministrazione dell’applicazione.
 \item Visualizzare la lista degli esercizi inseriti nella piattaforma.
\end{itemize}  & RDF16 \\ \hline

TV-RD14 & L’insegnante vuole visualizzare l’area per la gestione delle classi. All'insegnante è richiesto di:

\begin{itemize}
 \item Raggiungere la lista delle classi create;
 \item Visualizzare la vista di gestione di una propria classe. piattaforma.
\end{itemize}  & RDF14 \\ \hline

		  \rowcolor{LightGray}
TV-RP16 & Il moderatore vuole eliminare una segnalazione dalla relativa lista. Al moderatore è richiesto di:

\begin{itemize}
 \item Visualizzare la lista delle segnalazioni ricevute;
 \item Selezionare la segnalazione piattaforma;
 \item Selezionare l’opzione ”Elimina segnalazione”.
\end{itemize}  & RPF23 \\ \hline
TV-RP17 & Il moderatore vuole visualizzare la lista delle segnalazioni effettuate dagli utenti. Al moderatore è richiesto di:

\begin{itemize}
 \item Raggiunge la vista di amministrazione dell’applicazione;
 \item Selezionare l’opzione ”Lista delle segnalazioni”.
\end{itemize}  & RPF22 \\ \hline

		  \rowcolor{LightGray}
TV-RP18 & L’utente vuole segnalare un esercizio per abuso delle regole comportamentali. All'utente è richiesto di:

\begin{itemize}
 \item Raggiunge la vista di svolgimento di un esercizio che ritiene non conforme alle norme di comportamento;
 \item Selezionare l’opzione ”Segnala abuso”.
\end{itemize}  & RPF21 \\ \hline

TV-RO17 & L’utente vuole aggiungere una frase e svolgerla come esercizio. All'utente è richiesto di:

\begin{itemize}
 \item Raggiunge la vista principale dell’applicazione;
 \item Scrivere la frase da svolgere come esercizio;
 \item Confermare la frase indicata
\end{itemize}  & ROF20 \\ \hline

		  \rowcolor{LightGray}
TV-RD15 & L’utente vuole accedere alla pagina di registrazione alla piattaforma. All'utente è richiesto di:

\begin{itemize}
 \item Raggiunge la vista principale dell’applicazione;
 \item Selezionare l’opzione ”Registrati”.
\end{itemize}  & RDF21 \\ \hline

TV-RO18 & L’insegnante vuole visualizzare un messaggio di errore nel caso in cui stia inserendo una frase vuota come esercizio. All'insegnante è richiesto di:

\begin{itemize}
 \item Raggiunge la vista di inserimento di un nuovo esercizio e ha inserito;
 \item Visualizza un messaggio di errore ”La frase inserita è vuota”.
\end{itemize}  & ROF19 \\ \hline

		  \rowcolor{LightGray}
TV-RP19 & L’utente non riconosciuto vuole ricevere una e-mail per la conferma di iscrizione come insegnante. All'utente è richiesto di:

\begin{itemize}
 \item Inviare la richiesta di registrazione come insegnante;
 \item Riceve la conferma di registrazione come insegnante.
\end{itemize}  & RPF20 \\ \hline

		\caption{Test di validazione}
\end{longtable}

\subsection{Test di sistema}
Di seguito riportiamo una tabella contenente i test di sistema che intendiamo implementare per la verifica dei requisiti funzionali specificati nell'\textit{AnalisiDeiRequisiti\_v3.0.0}. \\
Per ogni test viene specificato il proprio codice univoco, il codice identificativo del requisito ad esso associato, la descrizione e lo stato di implementazione attuale.
	\begin{longtable}{|>{\centering\arraybackslash}m{1.6cm}|>{\centering\arraybackslash}m{1.7cm}|m{6.41cm}|}		
		\rowcolor{LightBlue}
		\textbf{\textcolor{white}{Codice\newline test}}
		& \textbf{\textcolor{white}{Codice\newline requisito}}
		& \multicolumn{1}{|c|}{\textbf{\textcolor{white}{ Descrizione}}}\\
		\hline
		\rowcolor{LightGray}
		TS-RO1
		& ROF32
		& Verifica che l'utente riesca a registrarsi alla piattaforma come allievo creando un account personale. 
		\\ \hline
		\rowcolor{white}
		TS-RO2
		& ROF2 
		& Verifica che l'utente possa eseguire l'accesso alla piattaforma utilizzando le sue credenziali.
		\\ \hline
		\rowcolor{LightGray}
		TS-RD1
		& RDF1 
		& Verifica che l'utente possa modificare i dati del proprio profilo personale.
		\\ \hline
		\rowcolor{white}
		TS-RD2
		& RDF2 
		& Verifica che l'amministratore possa verificare le credenziali di un utente che richiede la registrazione come insegnante. 
		\\ \hline
		\rowcolor{LightGray}
		TS-RD3
		& RDF3 
		& Verifica che l'utente venga avvisato in caso di errore nell'inserimento dei dati richiesti.
		\\ \hline
		\rowcolor{white}
		TS-RO3		
		& ROF3 
		& Verifica che l'insegnante e l'allievo possano ricercare degli esercizi sulla piattaforma.
		\\ \hline
		\rowcolor{LightGray}
		TS-RP1		
		& RPF1 
		& Verifica che durante la ricerca, l'utente abbia la possibilità di impostare dei filtri per raffinarla. 		
		\\ \hline
		\rowcolor{white}
		TS-RD4		
		& RDF4 
		& Verifica che dopo una ricerca, l'utente venga avvisato con un messaggio nel caso in cui il sistema non abbia trovato nessun risultato corrispondente ai criteri selezionati.
		\\ \hline
		\rowcolor{LightGray}
		TS-RP2		
		& RPF2 
		& Verifica che l'amministratore possa eliminare un utente iscritto alla piattaforma.
		\\ \hline
		\rowcolor{white}
		TS-RP3		
		& RPF3 
		& Verifica che l'insegnante possa modificare una soluzione di un esercizio da lui fornita
		\\ \hline
		\rowcolor{LightGray}
		TS-RD5		
		& RDF5 
		& Verifica che l'insegnante, accedendo alla sua area del profilo, possa visualizzare la lista degli esercizi da lui creati. 
		\\ \hline
		\rowcolor{white}
		TS-RP4		
		& RPF4 
		& Verifica che l'insegnante possa eliminare una soluzione di un esercizio da lui fornita. 
		\\ \hline
		\rowcolor{LightGray}
		TS-RP5		
		& RPF5 
		& Verifica che l'insegnante possa indicare gli argomenti trattati nell'esercizio in fase di creazione.
		\\ \hline
		\rowcolor{white}
		TS-RO4		
		& ROF4 
		& Verifica che l'allievo possa inserire una frase da svolgere o selezionare un esercizio da quelli disponibili sul sistema.
		\\ \hline
		\rowcolor{LightGray}
		TS-RO5		
		& ROF5 
		& Verifica che l'allievo possa compilare i campi relativi alle parole della frase al fine di completare l'esercizio selezionato.
		\\ \hline
		\rowcolor{white}
		TS-RD6		
		& RDF6 
		& Verifica che lo sviluppatore possa ricercare le annotazioni di una particolare frase.
		\\ \hline
		\rowcolor{LightGray}
		TS-RP6		
		& RPF6 
		& Verifica che lo sviluppatore possa ordinare la lista dei risultati ottenuti dalla ricerca tramite determinati parametri. 
		\\ \hline
		\rowcolor{white}
		TS-RP7		
		& RPF7 
		& Verifica che l'amministratore possa eliminare uno qualsiasi degli esercizi inseriti nel sistema.
		\\ \hline
		\rowcolor{LightGray}
		TS-RO6		
		& ROF6 
		& Verifica che l'insegnante possa inserire un esercizio nel sistema, indicando una soluzione per esso. 
		\\ \hline
		\rowcolor{white}
		TS-RO7		
		& ROF7 
		& Verifica che l'insegnante possa inserire la soluzione dell'esercizio che sta creando; può renderla pubblica o privata. 
		\\ \hline
		\rowcolor{LightGray}
		TS-RO8		
		& ROF8 
		& Verifica che l'allievo possa svolgere un esercizio da lui indicato e visualizzarne la relativa valutazione. 
		\\ \hline
		\rowcolor{white}
		TS-RD7		
		& RDF7 
		& Verifica che l'allievo, accedendo al proprio profilo, possa visualizzare dei dati relativi ai propri progressi. 
		\\ \hline
		\rowcolor{LightGray}
		TS-RD8		
		& RDF8 
		& Verifica che lo sviluppatore possa filtrare i dati trovati durante la ricerca ottenendo una lista di esercizi. 
		\\ \hline
		\rowcolor{white}
		TS-RD9
		& RDF9 
		& Verifica che lo sviluppatore possa impostare un filtro temporale per la ricerca degli esercizi.
		\\ \hline
		\rowcolor{LightGray}
		TS-RD10		
		& RDF10 
		& Verifica che lo sviluppatore possa includere o escludere dalla ricerca uno o più utenti. 
		\\ \hline 
		\rowcolor{white}
		TS-RP8		
		& RPF8 
		& Verifica che lo sviluppatore possa visualizzare i dati relativi ad una particolare annotazione. 		
		\\ \hline
		\rowcolor{LightGray}
		TS-RP9		
		& RPF9 
		& Verifica che lo sviluppatore possa visualizzare lo storico delle annotazioni. 
		\\ \hline
		\rowcolor{white}
		TS-RO9
		& ROF9 
		& Verifica che lo sviluppatore deve poter scaricare un file contenente i dati relativi agli esercizi ottenuti con la ricerca.
		\\ \hline
		\rowcolor{LightGray}
		TS-RP10		
		& RPF10 
		& Verifica che lo sviluppatore può visualizzare le informazioni relative ad uno dei modelli disponibili. 
		\\ \hline
		\rowcolor{white}
		TS-RP11		
		& RPF11 
		& Verifica che lo sviluppatore deve poter scaricare le informazioni riguardanti un modello. 
		\\ \hline
		\rowcolor{LightGray}
		TS-RP12		
		& RPF12 
		& Verifica che lo sviluppatore deve poter creare un modello tramite la piattaforma. 
		\\ \hline	
		
		\rowcolor{white}
		TS-RO10	
		& ROF10 
		& Verifica che l'insegnante possa creare una nuova classe. 
		\\ \hline
		\rowcolor{LightGray}
		TS-RO11	
		& ROF11 
		& Verifica che l'insegnante possa eliminare una classe dal sistema. 
		\\ \hline
		\rowcolor{white}
		TS-RO12
		& ROF12 
		& Verifica che l'insegnante possa aggiungere degli alunni ad una classe. 
		\\ \hline
		\rowcolor{LightGray}
		TS-RO13
		& ROF13 
		& Verifica che l'insegnante possa aggiungere degli esercizi a quelli assegnati ad una classe. 
		\\ \hline
		\rowcolor{white}
		TS-RP13
		& RPF13 
		& Verifica che l'insegnante possa visualizzare i progressi degli alunni di una propria classe.
		\\ \hline
		\rowcolor{LightGray}
		TS-RO14
		& ROF14 
		& Verifica che l'insegnante possa eliminare un alunno dalla lista di quelli iscritti ad una delle proprie classi. 
		\\ \hline
		\rowcolor{white}
		TS-RO15	
		& ROF15 
		& Verifica che l'insegnante possa visualizzare la lista degli alunni iscritti ad una delle sue classi. 
		\\ \hline
		\rowcolor{LightGray}
		TS-RO16
		& ROF16 
		& Verifica che l'utente possa visualizzare la lista delle proprie classi. 
		\\ \hline
		\rowcolor{white}
		TS-RD11	
		& RDF11 
		& Verifica che l'utente possa confermare le operazioni.
		\\ \hline
		\rowcolor{LightGray}
		TS-RO17
		& ROF22 
		& Verifica che l'utente riesca a registrarsi alla piattaforma come insegnante creando un account personale. 
		\\ \hline
		
		\caption{Test di sistema}
\end{longtable}

