\section{Introduzione}
\subsection{Scopo del documento}
Lo scopo del seguente documento è quello di garantire la qualità$^*$ del prodotto software e di conseguenza dei processi$^*$ impiegati per la sua realizzazione. Questo documento sarà composto in maniera incrementale, verrà modificato e migliorato durante l'arco temporale del progetto$^*$, in modo coerente al suo ciclo di vita. Nelle prossime pagine verranno quindi illustrate le strategie di verifica$^*$ e di validazione da utilizzare per tutto lo svolgimento del progetto. Queste tecniche serviranno ad assicurare che le attività vengano eseguite correttamente, individuando e correggendo tempestivamente eventuali errori.
\subsection{Scopo del prodotto}
Il prodotto ha lo scopo di realizzare una piattaforma collaborativa di raccolta dati.  La piattaforma consentirà a degli allievi di svolgere esercizi di analisi grammaticale proposti dai loro insegnanti.
Lo scopo finale della piattaforma è raccogliere dati relativi agli esercizi predisposti e al loro svolgimento, con il fine di utilizzare tali informazioni per insegnare l'analisi grammaticale ad un elaboratore; tramite tecniche di apprendimento automatico.\\
Le tipologie di utenti che usufruiranno di questo prodotto saranno:
\begin{itemize}
	\item insegnanti: prepareranno in modo semplice e rapido degli esercizi da somministrare ai propri allievi;
	\item allievi: eseguiranno gli esercizi ottenendo una valutazione immediata;
	\item sviluppatori: potranno accedere ai dati raccolti mediante la piattaforma per automatizzare il servizio.
\end{itemize}
\subsection{Note esplicative}
Tutti i termini marcati con un * ad apice presenti nel seguente documento trovano una definizione all'interno del \textit{Glossario}.\\
I nomi dei documenti interni/esterni prodotti dal gruppo OttoBit saranno scritti in corsivo.
\subsection{Riferimenti}
\subsubsection{Normativi}
\begin{itemize}
\item \textit{NormeDiProgetto\_v2.0.0}
\item \textit{AnalisiDeiRequisiti\_v2.0.0}
\item Capitolato d'appalto Colletta: piattaforma raccolta dati di analisi di testo\footnote{\url{https://www.math.unipd.it/~tullio/IS-1/2018/Progetto/C2.pdf}}
\end{itemize}
\subsubsection{Informativi}
\begin{itemize}
	\item K. EL EMAM, A. BIRK, \textit{Validating the ISO/IEC 15504 measures of software development process capability}, in \enquote{The Journal of Systems and Software}, 51 (2000), pp. 119-149\footnote{\url{https://www.uio.no/studier/emner/matnat/ifi/INF5181/h11/undervisningsmateriale/reading-materials/Lecture-11/SPICE/elemam-jss2000.pdf}}
	\item ISO/IEC 15504:2006\footnote{\url{http://artemisa.unicauca.edu.co/~cardila/CS\_10c\_ISO\_15504-5\_\_2006.PDF}}
	\item ISO/IEC 9126\footnote{\url{https://en.wikipedia.org/wiki/ISO/IEC_9126}}
\end{itemize} 
