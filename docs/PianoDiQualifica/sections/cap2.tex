\section{Obiettivi di qualità}
In questo capitolo vengono definiti gli obiettivi che il gruppo OttoBit si impegna a raggiungere in termini di qualità di processo e di prodotto, garantendoli nella realizzazione della piattaforma "Colletta". Ad ogni obiettivo sono associate una o più metriche$^*$ che permetteranno di effettuare una valutazione quantitativa. \\
Sia gli obiettivi che le metriche sono identificati univocamente da un codice alfanumerico, il quale permetterà di tracciarli facilmente e di controllarli costantemente. Le metriche adottate sono definite nelle NormeDiProgetto\_v2.0.0.

\subsection{Classificazione degli obiettivi}
La classificazione degli obiettivi rispetta la seguente notazione:
\begin{center}
	O[Ambito][Codice identificativo]
\end{center}

Dove:
\begin{itemize}
	\item Ambito: indica se l'obiettivo si riferisce a processi, prodotto documento oppure prodotto software, e può assumere i seguenti valori:
	\begin{itemize}
		\item PC: per indicare un obiettivo per il processo
		\item PD: per indicare un obiettivo per il documento
		\item PS: per indicare un obiettivo per il software
	\end{itemize}
	
	\item Codice Identificativo: intero incrementale a partire da 1.
	
\end{itemize}

\subsection{Processo}
Al fine di definire degli obiettivi di qualità di processo, il gruppo ha deciso di adottare lo standard ISO/IEC 15504 - noto anche come SPICE$^*$. Questo standard fornisce gli strumenti utili a valutare la qualità di processo e verrà utilizzato in associazione al ciclo di Deming o PDCA$^*$. Tale ciclo definisce un metodo di controllo mirato al miglioramento continuo del livello di qualità di processo evitando possibili regressioni. 
Vengono riportate le descrizioni dello standard e del ciclo, rispettivamente, nell'appendice C e sottosezione C.1. 
L'uso del ciclo di Deming in aggiunta allo standard SPICE permetterà di garantire:
\begin{itemize}
	\item monitorare e misurare costantemente le performance dei processi;
	\item perseguire un miglioramento continuo dei processi stessi.
\end{itemize}

\subsection{Prodotto}
Lo standard adottato per definire e delineare gli obiettivi è lo standard ISO/IEC 9126:200, il quale fornisce i criteri di applicazione delle metriche descritte BOH per verificare il livello di soddisfacimento degli obiettivi riportati nel capitolo 2. 
Di questo standard, viene riportata una descrizione nell'appendice D.
Man mano che si proseguirà con lo svolgimento del progetto, verranno realizzate due tipologie di prodotto, le quali dovranno rispettare determinati requisiti:
\begin{itemize}
	\item Documenti: devono essere leggibili, comprensibili e corretti;
	\item Software: 
	\begin{itemize}
		\item la piattaforma "Colletta" rispetta i requisiti obbligatori, desiderabili ed opzionali al BOH, BOH e BOH rispettivamente;
		\item la piattaforma "Colletta" supera i test nel BOH dei casi;
		\item il codice di "Colletta" è comprensibile, ben commentato e manutenibile.
	\end{itemize}
\end{itemize}

\subsection{Tabella degli obiettivi}

\begin{longtable}{| c | p{5cm} | p{5cm} | c |}
	\rowcolor{LightBlue}
	\color{white}\bfseries ID & \color{white}\bfseries Obiettivo & \color{white}\bfseries Descrizione & \color{white}\bfseries Metrica \\[0.25cm]
	OPC1 & Miglioramento continuo & Attività incessante e continua di miglioramento della performance dei processi. & MPC1 \\ \hline
	OPC2 & Stabilità dei requisiti & Capacità di individuare tutti e soli i requisisti adeguati. & MPC2 \\ \hline
	OPC3 & Qualità del sorgente & Efficienza nell'uso delle norme stilistiche di codifica. & MPC3 \\ \hline
	OPC4 & Implementazione della verifica & Quantificazione dei test per la verifica implementati. & MPC4 MPC5 \\ \hline
	OPD1 & Leggibilità dei documenti & I documenti devono assicurare un buon livello di leggibilità. & MPD1 \\ \hline
	OPS1 & Implementazione requisiti obbligatori & I requisiti obbligatori devono essere tutti implementati. & MPS \\ \hline
	OPS2 & Implementazione requisiti desiderabili & I requisiti desiderabili devono essere implementati al BOH. & MPS \\ \hline
	OPS3 & Copertura del codice & Almeno il BOH del codice deve essere coperto da test. & MPS \\ \hline
	OPS4 & Superamento dei test & Il codice deve superare almeno il BOH dei test. & MPS \\ \hline
	OPS5 & Manutenibilità del codice & Il prodotto software deve poter essere espanso o modificato agevolmente. & MPS \\ \hline 
	\caption{Tabella degli obiettivi} 
\end{longtable}