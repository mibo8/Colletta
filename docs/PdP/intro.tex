\subsection{Scopo del documento}
	Il documento ha lo scopo di definire la pianificazione del progetto ``Colletta: piattaforma raccolta dati di analisi di testo" proposto da MIVOQ s.r.l. per il gruppo OttoBit. Il documento presenta:
	\begin{itemize}
		\item un'analisi dei rischi in cui è possibile incorrere;
		\item una breve analisi sul modello di sviluppo scelto;
		\item la pianificazione dei tempi e delle attività;
		\item l'assegnazione delle attività pianificate ai membri del team;
		\item una stima preventiva delle risorse;
		\item la rendicontazione delle risorse impiegate.
	\end{itemize}

\subsection{Scopo del prodotto}
	Il prodotto richiesto dalla proponente è una piattaforma che permetta la raccolta di dati in modo implicito tramite la risoluzione di esercizi. Tali dati devono essere utilizzati per addestrare un software di apprendimento automatico$^*$ già esistente che, a sua volta, deve essere in grado di fornire una soluzione agli esercizi proposti. L'obiettivo del prodotto potrà essere raggiunto tramite l'impiego di un database$^*$ che garantisca la permanenza dei dati, il software di apprendimento automatico e un'interfaccia (web o di un'applicazione mobile) che permetta l'interazione con gli utenti.

\subsection{Glossario}
	All'interno del documento è possibile trovare termini ambigui: in tal caso, tali termini possono essere trovati nel Glossario insieme alla relativa spiegazione. I termini del glossario vengono indicati con un * in apice.
	
\subsection{Riferimenti}
	\subsubsection{Normativi}
		\begin{itemize}
			\item Norme di Progetto v1.0.0;
			\item Capitolato d'appalto C2: Colletta\footnote{\url{https://www.math.unipd.it/~tullio/IS-1/2018/Progetto/C2.pdf}}
			\item Regolamento organigramma\footnote{\url{https://www.math.unipd.it/~tullio/IS-1/2018/Progetto/RO.html}}
		\end{itemize}
	\subsubsection{Informativi}
		\begin{itemize}
			\item ISO/IEC 12207:1995
			\item Slide della lezione T5\footnote{\url{https://www.math.unipd.it/~tullio/IS-1/2018/Dispense/L05.pdf}}
		\end{itemize}

\subsection{Scadenze scelte}
	Il gruppo OttoBit ha scelto di rispettare le seguenti scadenze:
	\begin{enumerate}
		\item Revisione dei Requisiti: 2019-01-21;
		\item Revisione di Progettazione: 2019-03-15;
		\item Revisione di Qualifica: 2019-04-19;
		\item Revisione di Accettazione: 2019-05-17.
	\end{enumerate}