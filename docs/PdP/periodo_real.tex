Nella fase 3 hanno luogo le seguenti attività:
\begin{itemize}
	\item normazione: modifiche alle \textit{Norme di progetto} secondo quanto segnalato alla Revisione di progettazione. Si procede poi con il suo incremento;
	\item pianificazione della qualifica: modifiche al \textit{Piano di qualifica} secondo quanto segnalato alla Revisione di progettazione. Si procede poi con il suo incremento;
	\item pianificazione delle attività: modifiche al \textit{Piano di progetto} secondo quanto segnalato alla Revisione di progettazione;
	\item progettazione in dettaglio e Product Baseline: questa attività presenta la baseline architetturale del prodotto tramite diagrammi delle classi e di sequenza, mostrando coerenza con quanto mostrato nel PoC;
	\item codifica: realizzazione del prodotto;
	\item verifica per il colloquio: verifica del codice scritto in vista del colloquio Agile con il committente;
	\item colloquio: viene effettuato il colloquio con il committente;
	\item redazione manuali: redazione dei manuali utente e sviluppatore;
	\item incremento progettazione e codifica: in base a alle segnalazioni ricevute al colloquio con il committente viene eseguito l'eventuale incremento;
	\item incremento della pianificazione delle attività: viene aggiornato il \textit{Piano di progetto} con il consuntivo pre-finale;
	\item verifica per la consegna: vengono verificati tutti i documenti e il Product Baseline con la relativa codifica;
	\item consegna del materiale in ingresso;
	\item preparazione alla presentazione.
\end{itemize}