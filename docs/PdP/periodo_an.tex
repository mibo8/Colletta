	\subsubsection{Periodo di Avvio (2018-12-04 - 2018-12-17)}
		Nel periodo di avvio hanno luogo le seguenti attività:
		\begin{enumerate}[label = 1.\arabic*)]
			\item ricerca degli strumenti (2018-12-4 - 2018-12-14): tutti i membri del gruppo effettuano le ricerche sui possibili strumenti utili alle attività di avvio e di analisi dei requisiti;
			\item prima normazione (2018-12-5 - 2018-12-14): gli amministratori redigono le \textit{Norme di Progetto} concordate per i processi di supporto e organizzativi;
			\item studio di fattibilità (2018-12-4 - 2018-12-8): gli analisti effettuano lo studio di fattibilità dei capitolati;
			\item pianificazione di progetto (2018-12-7 - 2018-12-13): il responsabile redige il \textit{Piano di Progetto}, riportando modello di sviluppo, analisi dei rischi e la pianificazione per le prime attività dell'analisi dei requisiti;
			\item verifica dei documenti (2018-12-15 - 2018-12-16): i verificatori controllano se i documenti siano corretti.
		\end{enumerate}
		
	\subsubsection{Periodo di Analisi (2018-12-17 - 2018-02-09)}	
		\paragraph{Pianificazione (2018-12-17 - 2018-12-23)\\} Il periodo di analisi dei requisiti inizia con le attività di:
			\begin{enumerate}[label = 2.1.\arabic*)]
				\item pianificazione di progetto (2018-12-17 - 2018-12-21): il responsabile effettua il resoconto del periodo di avvio e pianifica in maniera più dettagliata le attività di analisi dei requisiti; 
				\item pianificazione della qualifica (2018-12-17 - 2018-12-21): i verificatori redigono il resoconto del periodo di avvio e gli amministratori effettuano i primi incrementi per il \textit{Piano di Qualifica};
				\item normazione (2018-12-17 - 2018-12-21): gli amministratori redigono in maniera precisa e completa le \textit{Norme di Progetto} per l'attività di analisi;
				\item verifica del \textit{Piano di Progetto}, del \textit{Piano di Qualifica} e delle \textit{Norme di Progetto} (2018-12-21 - 2018-12-23).
			\end{enumerate}
		\paragraph{Analisi dei requisiti (2018-12-19 - 2018-02-09)\\} La parte centrale del periodo di analisi dei requisiti è costituita dalle attività di:
			\begin{enumerate}[label = 2.2.\arabic*)]
				\item analisi dei requisiti del sistema (2018-12-19 - 2018-12-27): gli analisti svolgono la prima analisi dei requisiti del sistema;
				\item incrementi al \textit{Piano di Qualifica} (2018-12-26 - 2018-12-30): gli analisti introducono nel piano i test di sistema in base a quanto scaturito dall'analisi dei requisiti;
				\item verifica dell'analisi dei requisiti di sistema (2018-12-29 - 2018-12-31)
				\item verifica del \textit{Piano di Qualifica} (2018-01-09 - 2019-01-03)
				\item incrementi all'analisi dei requisiti (2019-01-02 - 2019-01-08): gli analisti aggiungono degli incrementi all'analisi di sistema
				\item verifica degli incrementi all'analisi dei requisiti (2019-01-09 - 2019-01-11)
				\item incrementi al \textit{Piano di Progetto} (2019-01-07 - 2019-01-10): il responsabile redige la parte di rendicontazione e consuntivo del \textit{Piano di Progetto} da presentare alla Revisione dei Requisiti.
				\item incrementi al \textit{Piano di Qualifica} (2019-01-07 - 2019-01-10): i verificatori redigono la parte di rendicontazione del \textit{Piano di Qualifica} da presentare alla Revisione dei Requisiti, aggiungendo i risultati delle misurazioni effettuate, e gli analisti aggiungono i restanti test di sistema.
				\item verifica del \textit{Piano di Progetto} e del \textit{Piano di Qualifica} (2019-01-11 - 2019-01-12)
				\item preparazione alla presentazione (2019-01-15 - 2019-01-20);
				\item analisi dei requisiti software (2019-02-01 - 2019-02-05): gli analisti effettuano l'analisi dei requisiti software. Quest'attività è successiva al primo incremento di progettazione architetturale.
				\item verifica dell'analisi dei requisiti (2019-02-07 - 2019-02-09)				
			\end{enumerate}

	\begin{table}[h]
		\caption{Tabella delle assegnazioni per il periodo di avvio}
		\centering		
		\begin{tabular}{| >{\centering}p{1.5cm} | c | c | c | c | c | c | c |}
			\rowcolor{LightBlue}
			\textbf{\color{white}Numero attività} 
			& \textbf{\color{white}BER} 
			& \textbf{\color{white}BOR} 
			& \textbf{\color{white}COS} 
			& \textbf{\color{white}MAR} 
			& \textbf{\color{white}PEA} 
			& \textbf{\color{white}PER} 
			& \textbf{\color{white}PET}\\

			1.1 & $\bullet$ & $\bullet$ & $\bullet$ & $\bullet$ & $\bullet$ & $\bullet$ & $\bullet$ \\
			\rowcolor{LightGray}
			1.2 & $\bullet$ &   &   & $\bullet$ & $\bullet$ &   & $\bullet$ \\
			1.3 & $\bullet$ &   & $\bullet$ & $\bullet$ & $\bullet$ & $\bullet$ & $\bullet$ \\ 
			\rowcolor{LightGray}
			1.4 &   &   & $\bullet$ &   &   &   &   \\ 
			1.5 &   & $\bullet$ &   &   & $\bullet$ & $\bullet$ &   \\ \hline
		\end{tabular}
	\end{table}

	\begin{table}[h]
		\caption{Tabella delle assegnazioni per il periodo di analisi}
		\centering
		\begin{tabular}{| >{\centering}p{1.5cm} | c | c | c | c | c | c | c |}
			\rowcolor{LightBlue}
			\textbf{\color{white}Numero attività} 
			& \textbf{\color{white}BER} 
			& \textbf{\color{white}BOR} 
			& \textbf{\color{white}COS} 
			& \textbf{\color{white}MAR} 
			& \textbf{\color{white}PEA} 
			& \textbf{\color{white}PER} 
			& \textbf{\color{white}PET}\\
		
			2.1.1  &   &   & $\bullet$ & $\bullet$ &   &   &   \\
			\rowcolor{LightGray}	
			2.1.2  &   & $\bullet$ &   &   &   & $\bullet$ &   \\
			2.1.3  & $\bullet$ &   &   &   & $\bullet$ &   & $\bullet$ \\
			\rowcolor{LightGray}
			2.1.4  & $\bullet$ & $\bullet$ &   &   &   & $\bullet$ &   \\
			2.2.1  &   &   & $\bullet$ & $\bullet$ & $\bullet$ & $\bullet$ &   \\
			\rowcolor{LightGray}
			2.2.2  &   &   &   & $\bullet$ & $\bullet$ & $\bullet$ &   \\
			2.2.3  & $\bullet$ & $\bullet$ &   &   &   &   & $\bullet$ \\
			\rowcolor{LightGray}
			2.2.4  &   &   & $\bullet$ &   &   &   & $\bullet$ \\
			2.2.5  & $\bullet$ & $\bullet$ &   &   &   & $\bullet$ & $\bullet$ \\
			\rowcolor{LightGray}
			2.2.6  &   &   & $\bullet$ & $\bullet$ & $\bullet$ &   &   \\
			2.2.7  &   &   &   &   & $\bullet$ &   &   \\
			\rowcolor{LightGray}
			2.2.8  &   &   & $\bullet$ & $\bullet$ &   &   &   \\
			2.2.9  & $\bullet$ &   &   &   &   &   & $\bullet$ \\
			\rowcolor{LightGray}
			2.2.10 & $\bullet$ & $\bullet$ & $\bullet$ & $\bullet$ & $\bullet$ & $\bullet$ & $\bullet$ \\
			2.2.11 & $\bullet$ & $\bullet$ & $\bullet$ &   &   &   &   \\
			\rowcolor{LightGray}
			2.2.12 &   &   &   & $\bullet$ & $\bullet$ &   &   \\ \hline
		\end{tabular}
	\end{table}
	
