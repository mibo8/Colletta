\subsection{Fase 1 (2018-12-04 - 2019-01-21)}
	\subsubsection{Ore impiegate}
		La seguente tabella indica le ore impiegate durante la fase 1. Tra parentesi viene indicata la differenza tra ore preventivate e ore impiegate ($preventivo - consuntivo$): valori positivi indicano le ore risparmiate e valori negativi indicano le ore in eccesso.
		\begin{table}[h]
			\centering
		\begin{tabular}{| l | c c c c c c | c |}
			\rowcolor{LightBlue}
			& \multicolumn{7}{c}{\textbf{\color{white}Numero di ore}}	\\
	
			\rowcolor{LightBlue}
			\textbf{\color{white}Membro}
			& \textbf{\color{white}RES}
			& \textbf{\color{white}AMM}
			& \textbf{\color{white}AN}
			& \textbf{\color{white}PRO}
			& \textbf{\color{white}DEV}
			& \textbf{\color{white}VER}
			& \textbf{\color{white}Totali}\\
	
			Bergo     		& -  (0)		& 9  (+6) 	& 18 (-4) 		& - (0) & - (0) & 7  (+1) 	& 34\\
			Bortone   		& -  (0)		& 14 (-4) 	& 14 (+1) 		& - (0) & - (0) & 10 (-2)	& 38\\
			Cosentino 		& 25 (-5) 	& -  (0) 	& 20 (+2) 		& - (0) & - (0) & -  (+5)	& 45\\
			Marcato   		& 10 (-5) 	& 10 (0) 	& 18 (+7) 		& - (0) & - (0) & -  (+3)	& 38\\
			Peagno    		& -  (+5) 	& 15 (0) 	& 19 (+3) 		& - (0) & - (0) & -  (+3)	& 34\\
			Peron     		& -  (0)		& 13 (-3) 	& 21 (+8) 		& - (0) & - (0) & 10 (-5)	& 44\\
			Pettenuzzo 	& - (0) 		& 12 (+3) 	& 5  (+14) 	& - (0) & - (0) & 10 (-5)	& 27\\ \hline
		\end{tabular}
	\end{table}
	\subsubsection{Costo}
		Le seguenti tabelle indicano i costi della fase 1. Nell'ultima colonna vengono indicate le differenze tra costi previsti e costi effettivi ($previsto - effettivo$): valori positivi indicano i risparmi e valori negativi indicano le perdite.
		\begin{table}[h]
			\centering
		\begin{tabular}{| l | l | l |}
			\rowcolor{LightBlue}
			\textbf{\color{white}Membro}
			& \textbf{\color{white}Costo}
			& \textbf{\color{white}Differenza}\\
			
			Bergo 				& 735€ 	& 35€\\
			Bortone 			& 780€ 	& -85€\\
			Cosentino 		& 1250€ 	& -25€\\
			Marcato 			& 950€ 	& +70€\\
			Peagno 			& 775€ 	& +270€\\
			Peron 				& 935€ 	& +65\\
			Pettenuzzo 	& 515€ 	& +335\\ \hline
			\textbf{Totale} & 5940€ & +665€\\ \hline
		\end{tabular}
		\end{table}
	\begin{table}[h]
		\centering
		\begin{tabular}{| l | l |l|}
			\rowcolor{LightBlue}
			\textbf{\color{white}Membro}
			& \textbf{\color{white}Costo}
			& \textbf{\color{white}Differenza}\\

			Responsabile 		& 1050€ 	& -150€\\
			Amministratore 	& 1460€ 	& +40€\\
			Analista 				& 2875€ 	& +775€\\
			Progettista 			& 0€ 		& 0€\\
			Programmatore 		& 0€ 		& 0€\\
			Verificatore 		& 555€ 	& 0€\\ \hline
			\textbf{Totale} 	& 5940€ 	& +665€\\ \hline
		\end{tabular}
		\end{table}
	\subsubsection{Conclusioni}
		\paragraph{Costi\\}
			La cifra prevista per l'investimento era di \textbf{6605€} e vi è stato un risparmio di \textbf{665€}. I costi effettivi ammontano, quindi, a \textbf{5940€}. 
		\paragraph{Scostamenti\\}
			Gli scostamenti rispetto a quanto previsto sono considerevoli. Ciò è imputabile a una previsione troppo pessimista delle ore necessarie per ogni attività. Tuttavia, un altro motivo può essere individuato nel mancato rispetto delle assegnazioni delle attività. Ciò è stato causato dalla frenesia da cui si è fatto prendere il gruppo nella parte successiva al periodo natalizio. Infatti, tra il 2018-12-21 e il 2019-01-06 i tempi non sono stati rispettati a causa di problemi di comunicazione tra i membri del gruppo. In seguito, a queste considerazioni è stato aggiunto alla tabella in §2.1 il rischio G03.