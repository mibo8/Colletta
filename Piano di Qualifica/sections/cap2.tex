\section{Visione generale delle strategie di verifica}
\subsection{Obiettivi di qualità}
	Per garantire la qualità del prodotto e dei processi utilizzati per realizzarlo il team Ottobit si è proposto di fissare degli obiettivi da perseguire per tutta la durata del progetto.
	\subsubsection{Qualità dei processi}
	Per il conseguimento degli obiettivi riguardanti la qualità dei processi è stato deciso di adottare lo standard ISO/IEC 15504 anche detta SPICE* acronimo di Software Process Improvement and Capability Determination. Questo standard viene utilizzato per eseguire una valutazione concreta della qualità dei processi, inoltre permette la misurazione della capability dei processi. All'appendice A è presente una descrizione delle caratteristiche dello standard.
	\subsubsection{Qualità del prodotto}
		Per quanto riguarda la qualità del prodotto si è scelto, in comune accordo, di seguire una serie di normative facenti parte dello standard ISO/IEC 9126, che definisce un modello dei requisiti qualitativi del Prodotto. Un completo approfondimento sullo standard ISO/IEC 9126 si trova in appendice B.
\subsection{Organizzazione}
Per fare in modo che la qualità del prodotto finale rimanga entro i livelli prestabiliti, è necessario che ogni processo del progetto sia verificato prima di passare al successivo, al fine di intercettare eventuali errori ed evitarne la loro propagazione. 
... Scrivere che si farà fede alle regole dettate dagli standard spiegati sopra...

\subsection{Pianificazione strategica}
La strategia generale adottata è quella di automatizzare il più possibile le operazioni di verifica grazie all'utilizzo di strumenti volti allo scopo. L'obiettivo è avere un riscontro affidabile e misurabile che permetta di assicurare il grado di qualità stabilito precedentemente.  L'aspettativa è la riduzione del lavoro manuale permettendo così un'attività di verifica più semplice ed efficace.

\subsection{Responsabilità}
Il verificatore ha il compito accertare che ogni modifica al materiale che costituisce il prodotto (codice, documentazione), effettuate dalle altre figure, venga svolta nella maniera corretta e nel rispetto delle regole definite in questo documento. Il Responsabile di progetto ha il compito di coordinare tutte le attività di verifica e di fare da garante della corretta esecuzione di tali, al fine di mantenere la qualità del materiale prodotto durante il suo ciclo di vita. Una descrizione più approfondita dei ruoli di progetto è inclusa nel documento  \textit{Piano di progetto 1.0.0}.



\subsection{Risorse}
...scrivere le risorse che verranno consumate durante l'attività di verifica...
\subsubsection{Necessarie}
\subsubsection{Disponibili}
\subsection{Tecniche di verifica}
...scrivere dell'analisi statica che viene fatta per misurare la qualità della documentazione quindi correzione ortografica tramite correttore latex di texmaker e calcolare indice di gulpease per ogni docuento verificato...
..Poi il resoconto della verifica verrà segnato nell'apposita sezione di questo documento...
\subsubsection{Analisi statica}
\subsubsection{analisi dinamica}