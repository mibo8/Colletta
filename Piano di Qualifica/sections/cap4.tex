\section{Resoconto attività di verifica}
\begin{longtable}{p{3cm} p{4cm} p{5cm} p{2cm}}
	\rowcolor{LightBlue}
		  \textbf{\textcolor{white}{Data}}
		& \textbf{\textcolor{white}{Autore}}
		& \textbf{\textcolor{white}{Documento}} 
		& \textbf{\textcolor{white}{Versione}}\\

		2018-12-16
		& Giovanni Peron
		& Piano di Qualifica 
		& 0.0.3\\
		\rowcolor{LightGray}
	\multicolumn{4}{p{15.25cm}}{\textbf{Descrizione:}Nella tabella del'analisi dei rischi del capitolo 2 ci sono ripetizioni nelle righe R01 e T01 entrambe nella colonna rilevamento. Il contenuto della tabella risulta tagliato a fine pagina 4. Nel paragrafo 3.1 a riga 6 suggerisco di inserire Proof of Concept nel glossario. In tutto il documento rivedere il formato delle date secondo le norme di progetto. Per informazioni più dettagliate vedi i commenti scritti nel file relativo al documento.\\
	\textbf{Indice di Gullpease:}95}\\
		\hline
		2018-12-16
		& Giovanni Peron
		& Piano di Qualifica 
		& 0.0.3\\
		\rowcolor{LightGray}
	\multicolumn{4}{p{15.25cm}}{\textbf{Descrizione:}Nella tabella del'analisi dei rischi del capitolo 2 ci sono ripetizioni nelle righe R01 e T01 entrambe nella colonna rilevamento. Il contenuto della tabella risulta tagliato a fine pagina 4. Nel paragrafo 3.1 a riga 6 suggerisco di inserire Proof of Concept nel glossario. In tutto il documento rivedere il formato delle date secondo le norme di progetto. Per informazioni più dettagliate vedi i commenti scritti nel file relativo al documento.\\
	\textbf{Indice di Gullpease:} 95}
\end{longtable}
