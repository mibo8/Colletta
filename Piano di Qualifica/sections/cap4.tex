\section{Resoconto attività di verifica}
\subsection{Prodotto}
\subsubsection{Documentazione}
Nella tabella seguente vengono riportati i risultati delle verifiche eseguite sui documenti. Il resoconto contiene le verifiche sia dei documenti esterni, cioè utili al committente, sia interni, utili invece al team Ottobit.\\
\begin{longtable}{p{3cm} p{4cm} p{5cm} p{2cm}}
	\rowcolor{LightBlue}
		  \textbf{\textcolor{white}{Data}}
		& \textbf{\textcolor{white}{Autore}}
		& \textbf{\textcolor{white}{Documento}} 
		& \textbf{\textcolor{white}{Versione}}\\

		2018-12-16
		& Giovanni Peron
		& Piano di Progetto
		& 0.0.3\\
		\rowcolor{LightGray}
	\multicolumn{4}{p{15.25cm}}{\textbf{Descrizione:} Nella tabella del'analisi dei rischi del capitolo 2 ci sono ripetizioni nelle righe R01 e T01 entrambe nella colonna rilevamento. Il contenuto della tabella risulta tagliato a fine pagina 4. Nel paragrafo 3.1 a riga 6 suggerisco di inserire Proof of Concept nel glossario. In tutto il documento rivedere il formato delle date secondo le norme di progetto. Per informazioni più dettagliate vedi i commenti scritti nel file relativo al documento.}\\
	\rowcolor{LightGray}
	\multicolumn{4}{p{15.25cm}}{
	\textbf{Indice di Gullpease:} 95
	}\\
		\hline
		2018-12-16
		& Michele Bortone
		& Norme di Progetto
		& 0.0.3\\
		\rowcolor{LightGray}
	\multicolumn{4}{p{15.25cm}}{\textbf{Descrizione:} 
	Il correttore segnala alcuni errori ortografici.
	Sezione "Scopo del documento" incompleta.
	Sezione 2.1 incompleta.
	Nella sezione 4.1.3.5 è presente un errore nella descrizione delle norme riguardanti l'inserimento delle figure all'interno di un documento. Suggerisco di aggiungere l'obbligo di inserire una breve didascalia dell'immagine corrispondente.
	Nella sezione 5.1.4 è presente un errore riguardo i compiti di ciascun ruolo. Bisogna correggere il redattore dello sudio di fattibilità.
	}\\
	\hline
		2018-12-16
		& Michele Bortone
		& Studio di Fattibilità
		& 0.0.3\\
		\rowcolor{LightGray}
	\multicolumn{4}{p{15.25cm}}{\textbf{Descrizione:} 
	Il correttore segnala alcuni errori ortografici.
	Elenchi puntati non conformi alle \textit{Norme di Progetto}.
	}\\
	\rowcolor{LightGray}
	\multicolumn{4}{p{15.25cm}}{
	\textbf{Indice di Gullpease:} 58
	}
\end{longtable}
\subsection{Processi}
\subsubsection{Avvio}
mettere una tabella per valutare i processi in base agli attributi e metodi forniti da ISO/IEC 155504
\subsubsection{Analisi del Sistema}
mettere una tabella per valutare i processi in base agli attributi e metodi forniti da ISO/IEC 155504
\subsubsection{Analisi software e Progettazione}
mettere una tabella per valutare i processi in base agli attributi e metodi forniti da ISO/IEC 155504
\subsubsection{Realizzazione}
mettere una tabella per valutare i processi in base agli attributi e metodi forniti da ISO/IEC 155504
\subsubsection{Validazione}
mettere una tabella per valutare i processi in base agli attributi e metodi forniti da ISO/IEC 155504

